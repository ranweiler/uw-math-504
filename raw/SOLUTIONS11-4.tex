\input amstex
\input amssym.def
\loadmsbm
\nopagenumbers
\magnification=\magstep1
\centerline{\bf SOLUTIONS TO HOMEWORK \#5, DUE 11-4}
\bigskip
\noindent 1. (a)  Let $I$ be an ideal of $R$.  If $I$=0, then there is only the 0 map from $I$ to a $K$-vector space $V$, which extends to 0 on $R$, so assume that $i\ne0$ and let $i\in I$.  Then any $R$-module map $f$ from $I$ to a $K$-vector space $V$ sends $i$ to $iv$ fro some $v\in V$, and if $i,j\in I$ are sent to $iv,jw\in V$, then looking at the image of $ij$ we see that $v=w$.  Hence there is a fixed $v\in V$ with $f(x) = xv$ for all $x\in I$, and $f$ extends to the map from $R$ to $V$ sending $r$ to $rv$.  By Baer's Criterion, $V$ is in injective over $R$.

   (b) Look at the ideal $I=(x,y)$ generated by $x$ and $y$ in $R$ and let $f:I\rightarrow K'$ send a combination $xp+yq$ to the image of $q$ in $K'$, for $p,q\in R$.  As $xp = yq$ if and only if there is a polynomial $r$ with $p=yr,q=xr$ (by unique factorization in $R$, it follows that $f$ is well defined.  If $f$ extends to all of $R$, then $f(1)$ would have to be the image of $(xp+1)/y$ in $K'$ for some
$p\in R$; but then $f(x)=x(xp+1)/y\ne0$ in $K'$, a contradiction, since $y$ cannot divide either $x$ or $xp+1$ for any $p$.
\vskip .5in
\noindent 2. For the first part, look at the set of proper two-sided ideals; this is partially ordered by inclusion and the union of any chain of proper ideals is still proper, as each ideal in the chain excludes $1$ and so the union does also.  Hence there is a maximal proper two-sided ideal.  The argument for left ideals is the same, as a proper left ideal must also exclude $1$.
\vskip .5in
\noindent 3. Letting $f$ be an element of $D=\hom_R(S,S)$, we see that the kernel and image of $f$ are both submodules of $S$, whence both are either all of $S$ or 0.  Hence either $f=0$ or $f$ is both one-to-one and onto and admits a two-sided inverse $f^{-1}$, which also lies in $D$, and $D$ is a division ring.
\vskip .5in
\noindent 4. First look at the left ideals of $R$.  We know there is a maximal proper left ideal $I$, which admits a left ideal complement in $R$ by projectivity; this complement must be simple as a left $R$-module, by maximality of $I$.  Hence $R$ has at least one (nonzero) minimal left ideal.  Now look at the set of all collections $\{L_\alpha:\alpha\in A\}$ of left ideals in $R$ such that the sum
$\sum L_\alpha$ is direct.  Such collections are partially ordered by inclusion and the union of chain of such collections is another one, so there is a maximal such collection.  The sum of the ideals in it, if proper, lies in a maximal left ideal, which has a minimal complement as above; but then this ideal could be added to the maximal collection, a contradiction.  Hence the sum is all of $R$.  But the element $1\in R$ is the sum of finitely many elements, each from one ideal in the collection, whence the finitely many ideals so involved already have direct sum $R$, and $R$ is the direct sum of finitely many minimal left ideals.

It follows at once that $R$ satisfies the descending chain condition on left or two-sided ideals:  given the direct sum $R=\oplus{i=1}^n L_i$, any infinitely strictly descending chain of left ideals would give rise to such a chain either in $L_1$ or $R/L_1\cong \oplus_{i=2}^n L_i$, which is impossible by induction.  It follows that any nonempty set of left ideals or two-sided ideals in $R$ has a minimal element.  

Thus $R$ has at least one minimal two-sided ideal $I$, which has a left ideal complement $J$.  Writing 1 as $e+f$ where $e\in I,f\in J$, we see that the left $R$-submodules $Re,Rf = R(1-e)$ of $I,J$ already have sum $R$, whence $I=Re,J=R(1-e)$.  Then $eR(1-e)\subset I\cap J = 0$, whence
$R(1-e)\subset (1-e)R$ (since $R$ is also the direct sum of $eR$ and $(1-e)R$ and $e(ex) = ex$ for all $x\in R$).  It follows that $(1-e)R$ is a two-sided ideal of $R$, which does not contain $e$ (since $ey = 0$ for all $y\in (1-e)R$, and the intersection $(1-e)R\cap Re = 0$ by minimality of $Re$.  But the sum of $Re$ and $(1-e)R$, as a right ideal, must be all of $R$, whence finally $J = (1-e)R$ is a two-sided ideal complementary to $I$.  Looking at two-sided $R$-subideals of $J$, we find another minimal one, which again admits a two-sided complement, and so on; in this way we get a direct sum
$R_1\oplus R_2\oplus\cdots$ of minimal two-sided ideals of $R$.  This sum must terminate after finitely many steps with a decomposition of all of $R$, by the descending chain condition, so at least we get the decomposition claimed.  Writing 1 as $\sum e_i$ with $e_i\in R_i$, we check immediately by multiplication that $\sum_i e_i = 1, e_i e_j = 0$ if $i\ne j, e_i^2 = e_i$, and at last we are done.
\vskip .5in
\noindent 5. As in the first part of the last problem, write each $R_i$ as the direct sum of finitely many simple left ideals $L_{ij}$.  Given two such ideals, say $L_{i1},L_{i2}$, note first that the annihilator
$\{x\in R_i: xL{i,2} = 0\}$ of $L_{i2}$ is a proper two-sided ideal, so must be 0, and there is $x\in L{_i2}$ with $L_{i1}x\ne0$; but then $L_{i1}x$ is a nonzero submodule of $L_{i2}$, which must be all of $L_{i2}$.  Similarly, $\{y\in L_{i1}:yx = 0\}$ is a submodule of $L_{i1}$, which must be 0, so we conclude that $L{i1}\cong L_{i2}$, as desired.
\end