Now let $V$ be a (finite-dimensional) $G$-module with submodule $W$.  We now make a fundamental assumption on our basefield $K$, namely that it has characteristic 0, and show that $W$ necessarily has a complement in $V$.  To do this, let $W'$ be any vector space complement of $W$ in $V$ and let $\pi:V\rightarrow W$ be the linear map that is the identity on $W$ and 0 on $W'$.  Replacing $pi$ by $\pi' = (1/n) \sum_g g\pi g^{-1}$, where $n$ is the order of $G$, we see that $g\pi' g^{-1} = \pi'$ for all $g\in G$, so that $\pi'$ is a $G$-homomorphism, which is still the identity on $W$.  Its kernel will then be the $G$-module complement $U$ to $W$ that we are looking for.   Iterating this result we see that {\sl every $G$-module over a field of characteristic 0, or more generally of characteristic not dividing the order of $G$, is a direct sum of irreducible modules}; another way to express this result is to say that every $G$-module is {\sl completely reducible} or {\sl semisimple}.  This result, known as {\sl Maschke's Theorem}, fails over every field $K$ whose characteristic $p$ does divide the order of $G$.   To see this, note that $G$ permutes the basis elements $g$ of $KG$, we see that the sum $\sum k_g$ of the coefficients $k_g$ of any element of $KG$ is preserved by $G$, whence the subspace $S = \{v = \sum k_g g \in KG:  \sum k_g = 0\}$ is a $G$-submodule such that $G$ acts trivially on $KG/S; but one easily checks that the only elements $\sum k_g g$ in $G$ on which $G$ acts trivially have $k_g = k_h$ for all $g,h\in G$.  Thus if the characteristic of $K$ does not divide the order of $G$, the $K$-subspace spanned by $\sum g$ is a $G$-stable complement of $S$ in $KG$, but if the characteristic of $K$ does divide the order of $G$, then $S$ has no $G$-stable complement in $KG$.

Returning to the case where the characteristic of $K$ does not divide the order of $G$, we now know that the group ring $KG$ is semisimple as a left module over itself, whence all results from the last two problems of this week's homework apply to it:  $KG$ is the direct sum of finitely many minimal two-sided ideals $I$ and each $I$ is the direct sum of finitely many minimal left ideals $L_i$, any two them isomorphic (say to $L$) as left $I$-modules.  Now bring Schur's Lemma into the picture:  the ring $I'$ of $I$-homomorphisms from any $L_i$ to itself is a division ring $D$ that is independent of $i$.  Regard $D$ as acting on $L_i$ on the {\sl right}, so that the product $xy$ of two such homomorphisms $x,y$ is taken to be the composition of $x$ and $y$ in that order.  The ring $R$ of $I$-homomorphisms from all of $I$ to itself as a left module is then the ring $M_n(D)$ of $n\times n$ matrices over $D$, where $n$ is the number of minimal left ideals $L_i$.  This is because any  $I$-homomorphism $\pi$ is completely determined by the projection $p_j(\pi(L_i)$ of the image $\pi(L_i)$ to $L_j$ and this projection defines an isomorphism from $L$ to itself, which is given by an element of $D$.   But any left module homomorphism from $I$ to itself is given by {\sl right} multiplication by an element of $I$ (this is true of any ring).  Hence the ring $I'$ of all such homomorphisms is isomorphic to $I$ as an abelian group, but the product $xy$ of two elements of it equals the product $yx$ in $I$.  We deduce that {\sl each $I$ is isomorphic to the ring $M_n (D')$ of $n \times n$ matrices over $D'$, the division ring obtained from $D$ by replacing the product $xy$ of two elements of it by the product $yx$ in $D$.  Up to isomorphism, the only simple left $I$-module is $D^n$, the space of column vectors of length $n$ over $D$}.  This property holds of any ring $R$ such that every left $R$-module is projective; in that generality the above result is called the {\sl Artin-Wedderburn Theorem}.  In our current setting, we see that $KG$ must be a finite direct sum of such matrix rings $M_{n_i}(D_i)$, where in addition each division ring $D_i$ is finite-dimensional over $K$ with $K$ in its center.  If now we further assume that $K$ is algebraically closed, the only division ring $D_i$ with $K$ in its center that is finite-dimensional over $K$ is $K$ itself, since any element $x$ in such a ring is algebraic over $K$, whence it must lie in $K$.  We conclude that {\sl in particular, the complex group algebra $\Bbb CG$ of any finite group $G$ is a direct sum of matrix rings over
$\Bbb C$}.  A left module over such ring is just the column vectors over $\Bbb C$ of the same size as one of its matrix ring factors, withe the other matrix factors acting by 0.  Thus {\sl $G$ has only finitely many inequivalent irreducible representations and the sum of the squares of their dimensions equals the order of $G$} (since the only irreducible module over $M_n(\Bbb C)$ is $\Bbb C^n$ and the dimension of $M_n(\Bbb C}$ over $\Bbb C$ is $n^2$).  We will determine the number of inequivalent representations of $G$ later.  

For now let's look at another example.  Consider again the group $G$ of quaternion units.  We know from last time that $G$ has a two-dimensional representation over $\Bbb C$, namely the ring $H$ of quaternions, which one easily checks is irreducible.  Besides this $G$ has four inequivalent 1-dimensional representations, each trivially itrreducible; in all of them $-1\in G$ acts trivially, while each of $i$ and $j$ acts by $1$ or $-1$ independently.  As the order 8 of $G$ is the sum $4+1+1+1+1$ of the squares of the dimensions of the irreducible representations found so far, we have found all of them.  By way of an interesting contrast, look at the group algebra $\Bbb RG$ of this same group over the field $\Bbb R$ of real numbers.  This time $H$ remains irreducible, but is now four-dimensional over $\Bbb R$, while the remaining irreducible representations found above of course remain irreducible.  Thus $\Bbb RG$, instead of being a sum of matrix rings over $\Bbb R$ or $\Bbb C$, is sum of four $1\times 1$ matrix rings over $\Bbb R$ plus another such ring over $H$ (it turns out that $H$ is itself a division ring).  This division ring can arise in real group algebras, but never in complex ones.