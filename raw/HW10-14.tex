\input amstex
\input amssym.def
\loadmsbm
\nopagenumbers
\magnification=\magstep1
\centerline{\bf HOMEWORK \#2, DUE 10/14}
\bigskip
\centerline{Math 504A}
\bigskip
\noindent 1. Show that the free product $\Bbb Z_2\ast\Bbb Z_2$ of two copies of $\Bbb Z_2$ is isomorphic to the infinite dihedral group, that is the semidirect product of $\Bbb Z$and $\Bbb Z_2$, where the nontrivial element of $\Bbb Z_2$ acts on $\Bbb Z$ by sending any integer to its negative.
\vskip .5in
\noindent 2. Show that the group $P=PSL_2(\Bbb Z)$ is isomorphic to the free product of $\Bbb Z_3$ and $\Bbb Z_2$, as outlined below.

(a) Show that the matrices $A' = \pmatrix -1&-1\\ 1&0\\ \endpmatrix$ and 
$B' = \pmatrix 0&-1\\ 1&0\\ \endpmatrix$ generate the group $S=SL_2(\Bbb Z)$, by first showing that the matrix $C = \pmatrix 1&1\\0&1\\ \endpmatrix$ is generated by $A'$ and $B'$.  Then, given any matrix $M\in S$, show how to multiply $M$ on the left by products of suitable powers of $C$ and $B'$ to perform any desired row operation on it (preserving the determinant as 1) with integer coefficients.  Using the Euclidean algorithm, transform the first column of $M$ into $\pmatrix 1\\ 0\\ \endpmatrix$ by such operations, and then observe that $M$ must now be a power of $C$.

(b) It follows that the images $A,B$ in $P$ generate $P$; note that $A$ has order 3 while $B$ has order 2.  Now show that no nonempty product of elements in $P$ that are alternately $A$ or $A^2$ and $B$ can equal 1.  (Look at the linear fractional transformations $T_1,T_2,T_3$, corresponding to $A,A^2,B$, respectively, and observe that $T_1$ maps positive irrational numbers to negative irrational numbers less than $ -1$, $T_2$ maps positive irrational numbers to negative irrationals greater than $-1$, and finally that $T_3$ sends negative to positive irrationals.)  Deduce the desired result.
\vskip .5in
\noindent 3. Find products $X_1,X_2$ of $A,B$ corresponding to the matrices $\pmatrix 1&2\\ 0&1\\ \endpmatrix$ and $\pmatrix 1&0\\ 2&1\\ \endpmatrix$ in $P$ and show that the $X_i$ freely generate a subgroup of $P$ (which turns out to have finite index).
\vskip .5in
\noindent 4. Classify the subgroups of index two of the free group $F_2$ on two generators $x,y$, giving a set of free generators of each such subgroup.
\vskip .5in
\noindent 5. Find a subgroup of $F_2$ that is free on infinitely many generators and give the generators explicitly.
\end 