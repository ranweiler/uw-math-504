\input amstex
\input amssym.def
\loadmsbm
\nopagenumbers
\magnification=\magstep1
\centerline{\bf SOLUTIONS TO HOMEWORK \#2, 10-14}
\bigskip
\noindent 1. Let $x,y$ be generators of two copies of $\Bbb Z_2$ and let $G$ be the free product of these copies.  By definition of this product, the elements of $G$ are exactly the powers
$(xy)^n,(yx)^n=(xy)^{-n}$ for $n$ a nonnegative integer and the products $x(xy)^n$ for $n$ an arbitrary integer.  It follows at once that $xy$ has infinite order in $G, x$ has order 2, and the conjugate of $xy$ by $x$ is $yx=(xy)^{-1}$.  These properties are the ones defining the infinite dihedral group $D_{\infty}$, whence $G\cong D_{\infty}$, as required.
\vskip .5in
\noindent 2. (a)  Begin by noting that the matrix $C=B' (A')^2$, so does indeed lie in the subgroup $G$ of $SL_2(\Bbb Z)$ generated by $A'$ and $B'$.  Multiplying a matrix $M$ in $SL_2(\Bbb Z)$ on the left by $C^k$ amounts to adding $k$ times the second row of $M$ to its first row; multiplying $M$ on the left by $B'$ interchanges its two rows and then replaces the first row by its negative.  Now one step of the Euclidean algorithm, applied to a pair $(a,b)$ of integers not both 0,, replaces whichever of $a,b$ has the larger absolute value by its remainder on division by the other, while leaving the other integer unchanged.  Iterating this, we replace the original pair $(a,b)$ by $(c,0)$, where $c$ is (say the positive) greatest common divisor of $a$ and $b$.  Applying this algorithm to the entries $a,b$ in the first column of $M$ and changing signs as necessary, we can replace this column by the one with entries $(1,0)$, while the determinant of $M$ is still 1.  Then the entries of the second column of $M$ must be $k,1$ for some integer $k$, whence $M$ is now the $k$-th power $C^k$ of $C$.  Hence $G$
is all of $SL_2(\Bbb Z)$, as claimed.
 
 (b) It is immediate (as claimed in the problem statement) that the images $A,B$ of $A',B'$ in
 $PSL_2(\Bbb Z)$ have orders 3 and 2, respectively.  The linear fractional transformations
 $T_1,T_2$, and $T_3$ corresponding respectively to $A,A^2$, and $B$ send $z$ respectively to\hfil\linebreak
 $(-z-1)/z = -1-(1/z),1/(-z-1),-1/z$, whence indeed $T_1$ sends positive irrational numbers to negative ones less than $-1, T_2$ sends positive irrationals to negative ones greater than $-1$, and $T_3$ sends negative irrationals to positive ones.  Now let $w_1\ldots,w_k$ be a word of odd length whose letters are alternately $A$ or $A^2$ and $B$.  Conjugating it by $B$ if necessary we may assume that it starts and ends with $B$.  The corresponding product of $T_1,T_2,T_3$ then sends negative irrationals to positive ones, so cannot be the identity transformation.  Similarly, if instead
 $w_1\ldots w_k$ has even length $k$ but is nonempty, then by conjugation we may assume that it starts with $B$ and ends with $A$ or $A^2$.   Then the corresponding product of $T_1,T_2,T_3$ sends negative irrationals to negative irrationals less than $-1$ (if $w_k = A$) or to negative irrationals greater than $-1$ (if $w_k = A^2$), so cannot be the identity transformation in either case.  We conclude that $PSL_2(\Bbb Z)$ is the free product of its cyclic subgroups of orders 3,2 generated by
 $A,B$, respectively, as desired.
 \vskip .5in
 \noindent 3. Observe first that $C^2 = \pmatrix 1&2\\0&1\\ \endpmatrix = (BA^2)^2$ while similarly 
 $(C^t)^2 = (BA)^2$.  Examining products of powers of $(BA^2)^2$ and $(BA)^2$, we see that any such nonempty product reduces to a nonempty product of terms alternating between $A$ or $A^2$ and $B$, which is not the identity by the previous problem.  Hence the subgroup of $PSL_2(\Bbb Z)$ generated by $C^2$ and  $(C^t)^2$ is freely generated by these elements, both of them having infinite order.  Thus this subgroup is free on two generators, as desired.
 \vskip .5in
 \noindent 4. Given the free group $F_2$ on two generators $x,y$ it is immediate that the only possibilities (up to equivalence) for a Schreier transversal of a subgroup $S$ of index 2 are $\{1,x\}$ and $\{1,y\}$.  In the first case the element $y$ of $F_2$ lies either in the identity coset of $S$ or the coset of $x$; if the latter holds the coset of $yx$ must be the identity coset, since $S$ must be normal.  Applying the recipe in class for the free generators of a subgroup of a free group, we get just three possibilities for these generators, namely $\{x,y^2,yxy^{-1}\},\{y,x^2,xyx^{-1}\}$, or
 $\{x^2,yx^{-1},xy\}$ (note that there is some latitude in the choice of generators in all three cases).
 \vskip .5in
 \noindent 5. The easiest example of a subgroup of $F_2$ that is free on infinitely many generators (and thus necessarily of infinite index) is the normal subgroup generated by $x$.  Here a Schreier transversal consists of all the powers of the other variable $y$ and we get $\{y^i x y^{-i}: i\in\Bbb Z\}$ as a set of free generators of this subgroup.  We could also take the set of such elements
$y^i x y^{-i}$ with $i$ running through the nonnegative integers only as generators of a different free subgroup of $F_2$.
\end
 
 