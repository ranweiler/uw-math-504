Continuing with the setup we ended with last time, let $V$ as before be the irreducible representation of $G$ corresponding to $\pi:G\rightarrow GL_n(\Bbb C)$ and consider what happens when we average over $G$ the linear map $\mu$ from $\Bbb C^n$ to itself with matrix having 1 as its $ij$-entry and 0 everywhere else.  We must get a scalar map $f$ every time, but now it need not be 0.  If
$i\ne j$, then the trace of $f$ must be the same as that of $\mu$, which is 0, so $f=0$ and we deduce as before that $\sum_{g\in G} \pi(g)_{ii} \overline{\pi(g)_{jj}} = 0$; but now if $i=j$, then the trace of
$\mu$ is 1, and so $f$ must have matrix $(1/n)I$.  Taking traces and adding, we deduce that
$(1/ | G |)\sum_{g\in G} \chi_pi(g)\overline{\chi_pi(g)} = 1$.  To interpret this last relation and the one at the end of the last lecture, we recall some linear algebra over $\Bbb C$.  The vector space $\Bbb C^n$ has a standard {\sl Hermitian form}, which attaches to the pair of vectors
$v = (v_1,\lodts,v_n), w = (w_1,\ldots,w_n)$ the complex scalar $(v,w) = \sum v_i\overline{w_i}$.  Like the dot product on $\Bbb R^n, (v,w)$ is linear in the first variable $v$ and positive definite in the sense that $(v,v)$ is a nonnegative real number for all $v\in\Bbb C^n$ and 0 only for $v=0$, but a difference is that $(v,w)$ is conjugate linear in the second variable $w$, so that
$(v,\alpha w) = \overline{\alpha}(v,w)$ if $\v,w\in\Bbb C^n,\alpha\in\Bbb C$.  Now the set $C$ of complex-valued functions on $G$ that are constant on conjugacy classes is a finite-dimensional vector space over $\Bbb C$, of dimension equal to the number $m$ of conjugacy classes in $G$.  The above results on characters show that $\sl this space has an orthonormal set of $m$ vectors with respect to the Hermitian form $(c_1,c_2) = (1/ |G| )\sum_{g\in G} c_1(g)\overline{c_2(g))$, namely the characters $\chi_1,\ldots,\chi_m$ of the distinct irreducible representations of $G$}.  This famous result is known as {\sl Schur orthogonality of the characters}.  It implies that {\sl the $\chi_i$ form an orthonormal basis of $C$} (like any orthonormal set of $d$ vectors in a complex vector space of dimension $d$ with a Hermitian form).  In particular, any complex fuinction on $G$ that is constant on conjugacy classes is uniquely a combination of irreducible characters.  Moreover, as the trace of any matrix in block diagonal form with blocks $B_1,\ldots,B_r$ is the sum of the traces of the $B_i$,
{\sl any direct sum $V$ of irreducible representations $V_1,\ldots,V_r$ has as its character the sum of its characters on the $V_i$}.  In fact, we can compute the multiplicity of any irreducible representation with character $\chi_i$ in $V$ as the inner product $(1/|G|) \sum_g \chi(g)\overline{\chi_i(g)$.  In particular, the character of a representation determines the representation, so that (as promised above) the single number we are attaching to a matrix $\pi(g)$ captures all the information of
$\pi(g)$ itself.

The information contained in the characters $\chi_i$ is most conveniently organized in the form of the {\sl character table} of $G$.  This is a square matrix whose rows are indexed by irreducible representations $\pi_i$ of $G$ and whose columns are indexed by conjugacy classes $C_j$ in $G$; the $ij$-entry equals the common value of $\chi_i$ on every element of $C_j$, where $\chi_i$ is the character of $\pi_i$.  (So as not to repeat information, it is customary to list only conjugacy classes, not all of their elements, in the table; we can either describe a conjugacy class as a whole or just give a representative element.)  This matrix becomes one with orthonormal rows if its $ij$-entry is multiplied by $\sqrt{| C_j |/| G |}$, where $| C_j |$ is the size of the conjugacy class $C_j$.  But any
square matrix $M$ with orthonormal rows also has orthonormal columns (since orthonormality of both rows and columns is expressed by the single matrix condition $M \overline{M}^t = I$).  Hence we get a second set of orthogonality relations:  $\sum_{\chi} \chi(g)\overline{\chi{g'}} = 0$ if the elements $g,g'$ of $G$ are not conjugate, where the sum runs over all irreducible characters $\chi$, while
$\sum_{\chi} \chi(g)\overline{\chi(g')} = | G |/| C_g |$ if $g,g'$ are conjugate in $G$ and the conjugacy class $C_g$ of $g$ has size $| C_g |$. 

Let's look at some character tables.  I will leave it as an exercise to construct the character tables of some small abelian groups; here all the matrices of which one is computing the traces are
$1\times 1$, so it is obvious that the characters determine the representations, but the orthogonality of any two rows in the table is already an interesting property, which has been exploited for various combinatorial purposes.  For $G = S_3$, the symmetric group on three letters, there are just three irreducible representations:  the trivial, sign, and reflection representations, where the sign representation attaches 1 to an even permutation, $-1$ to an odd one, while the reflection representation (as mentioned above) comes from the action of $S_3$ on an equilateral triangle.  Listing the representations in this order from top to bottom, and the conjugacy classes as the identity, all 3-cycles, and all transposition from left to right, we find that the top row of the character table is
$1\,1\,1$, its second row is $1\,1\,-1$, while its bottom row is $2,\,-1\,0$.  In particular, all entries are integers; we will see later that this is true of the character table of any symmetric group $S_n$. 