\input amstex
\input amssym.def
\loadmsbm
\nopagenumbers
\magnification=\magstep1
\centerline{\bf HOMEWORK \#3, DUE 10/21}
\bigskip
\centerline{\bf MATH 504A}
\bigskip
\noindent 1. Let $M$ be an $n\times n$ matrix over a commutative ring $R$ with $\det M = 0$.  Show that there is a nonzero $v\in R^n$ with $Mv=0$, by first letting $k$ be the largest positive integer (if  any) with some $k\times k$ submatrix of $M$ having nonzero determinant, and using determinants of suitable $k\times k$ submatrices of $M$ as the coordinates of $v$.  Deduce that no $R$-module map from $R^n$ to $R^m$ can be injective if $n>m$.
\vskip .5in
\noindent 2. Show that the ring $R$ of linear transformations from the direct sum $\Bbb R^\infty$ of countably many copies of the real numbers $\Bbb R$ to itself is such that $R\cong R\oplus R$ as an $R$-module, by dividing a basis for the domain of any such transformation into two countably infinite subsets.
\vskip .5in
\noindent 3. Show that the direct {\sl product} $M=\Bbb Z^\omega$ of countably many copies of $\Bbb Z$, consisting by definition of all sequences $(z_1,z_2,\ldots)$ with the $z_i\in\Bbb Z$ but no other restriction is not a free $\Bbb Z$-module, as follows.  First note that $M$ is uncountable, while the direct sum $N=\Bbb Z^\infty$ consisting of all sequences with all but finitely many $z_i$ equal to 0 is countable.  If $M$ had a basis $B$ over $\Bbb Z$, then some countable subset of $B$, say $B'$, would span $N$.  Let $M'$ be the quotient of $M$ by the span of $B'$; then $M'$ would be free with basis the images in it of the elements in $B$ but not $B'$.  The span of $B'$ is countable, so at least one of the uncountably many elements $(\pm 1!,\pm 2!,\ldots)$ has nonzero image $v$ in $M'$.  Show that for any nonzero integer $i$ there is $v_i\in M'$ with $iv_i = v$; but no nonzero element of a free
$\Bbb Z$-module has this property.
\vskip .5in
\noindent 4. Classify the finitely generated $\Bbb Z$-submodules of $\Bbb Q$ and show in particular that the {\sl subring} of $\Bbb Q$ generated by $1/2$ is not finitely generated as a $\Bbb Z$-module.
\vskip .5in
\noindent 5. Show that the tensor product (over $\Bbb Z$) of the $\Bbb Z$-modules $\Bbb Z_m$ and $\Bbb Z_n$ is 0 whenever $m,n$ are relatively prime integers.
\end 
 