Let $G$ be a group, $H$ a subgroup of $G$.  Let $X$ be a (left) transversal of $H$ in $G$, so that by definition $X$ consists of exactly one element of every left coset of $H$ in $G$.  Then every element of $G$ may be uniquely written as $xh$ for some $xi\in X,h\in H$.  If $x\in X,g\in G$ write 
$gx = \overline{gx}a_{g,x}$, where $\overline{gx}\in X,a_{g,x}\in H$.  It follows at once that if the set $S$ generates $G$ as a group, then the $a_{s,x}$ and $a_{s^{-1},x}$ generate $H$, as $s$ runs over $S$ and $x$ over $X$.  This already has an interesting consequence:  {\sl any subgroup of finite index in a finitely generated group is finitely generated}.  (This fact is interesting because, as you will see in homework next week, subgroups of finitely generated groups need not be finitely generated in general.  But now we easily check that $\overline{s^{-1}\overline{sx}}= x$ for any $s\in S,x\in X$, whence $a_{s^{-1},\overline{sx}} = a_{s.x}^{-1}$.  Hence the $a_{s,x}$ for $s\in S$ and $x\in X$ already generate $H$, and {\sl a subgroup of index $j$ of a group generated by $m$ elements is generated by $jm$ elements}.  Now it turns out that even this result can be improved, but to do so we need to look first at a very special class of groups.  Let $S=\{s_i:i\in I\}$ be any set and let $F$ be the free group on $S$:  by definition $F$ consists of all {\sl reduced words} $x_1\ldots x_m$, where each $x_i$ is either $s_j$ or $s_j^{-1}$ for some $j\in I$ and no $s_j$ appears next to $s_j^{-1}$ in
$x_1\ldots,x_m$.  We multiply two reduced words $x_1\ldots x,m,y_1\ldots y_n$ by starting with the concatenation $x_1\ldots x_m y_1\ldots y_n$ and then crossing out consecutive terms $s_j s_j^{-1}$ or $s_j^{-1} s_j$ until a reduced word is obtained; we will take the uniqueness of the reduced word we produce as obvious, though some books write out a proof of this in detail.  The identity in $F$ is then the empty word, which we denote $1$.  A reduced word $x_1\ldots x_n$ is said to have length $n$.

Letting $H$ be any subgroup of a free group $F$ we now choose a transversal $X$ in a special way, so that if a reduced word $x_1\ldots x_k$ lies in $X$, then so does $x_i\ldots x_k$ for all $i$ between 1 and $k$.  We do this inductively:  starting with the partial transversal consisting of the empty word $1$ alone, we suppose inductively that we have found a partial Schreier transversal the meets every left coset of $H$ represented by a word of length at most $k-1$, and then let $C=x_1\ldots x_kH$ be a typical left coset of $H$ represented by a word of length $k$.  If $X$ does not already meet this coset, then it meets the coset $x_2\ldots x_kH$, say in the word $w_1\ldots w_m$.  Add the word $sw_1\ldots w_m$ to $X$, so that it now meets $C$; this word must be reduced since otherwise $C$ would be represented by a word of length at most $k-1$ and thus a representative of it would appear in $X$.  Repeating for all words of length $k$, we extend $X$ so that it now meets every coset represented by any such word; by iteration we finally arrive at a transversal $X$ with the desired property.  We call such a transversal a {\sl Schreier transversal}.  Now I claim that {\sl if $X$ is a Schreier transversal of $H$, then the elements $a_{s,x}$ different from 1 freely generate $H$}.  To see this note first that a typical element $a_{s,x} = \overline{sx}^{-1} sx$ is equal to $w_m^{-1}\ldots w_1^{-1} s v_1\ldots v_n$, where $v_1\ldots v_n$ is a reduced word for $x$ and $w_1\ldots w_m$ a reduced word for $\overline{sx}$.  The reduced word $v_1\ldots v_n$ for $x$ cannot begin with $s^{-1}$, lest $sx = v_2\ldots v_n$ lie in $X$ since $X$ is Schreier, forcing $a_{s,x} = 1$.  Similarly, the reduced word $w_1\ldots w_m$ for $\overline{sx}$ cannot begin with $s$, lest $w_2\ldots w_m$ lie in $X$ and again $a_{s,x} = 1$.  Hence the word $w_m^{-1}\ldots w_1^{-1} s v_1\ldots v_n$ for $\overline{sx}$ is reduced.  In a similar manner, if we concatenate reduced words for various $a_{s,x}$ and
 $a_{s,x}^{-1}$ (with no $a_{s,x}$ appearing next to an $a_{s,x}^{-1}$) then no \lq\lq tail" $s v_1\ldots v_n$ can be fully cancelled by the next \lq\lq head" $w_m^{-1}\ldots w_1^{-1}$, lest the corresponding $sx = w_i\ldots w_m$ lie in $X$, implying that the $a_{s,x}$ term with tail $s v_1\ldots v_n$ was equal to 1; similarly for $a_{s^{-1},x}$ terms.  Thus any reduced product of nonidentity $a_{s,x}$'s and $a_{s',x'}^{-1}$'s has the same head in its reduced word as its first term does, so is not the identity.   The nonidentity $a_{s,x}$ freely generate $H$.
 
 How many generators $a_{s,x}$ are equal to 1?  We have $a_{s.x} = 1$ if and only if $sx\in X$; if this happens then either the reduced word for $sx$ begins with $s$ or not.  If it does, then the reduced word for $x$ cannot begin with $s^{-1}$; if it does not, then the reduced word for $x$ must have begun with $s^{-1}$ (so that indeed the reduced word for $sx$ does not begin with $s$).  The upshot is that we get a generator $a_{s,x}$ equal to 1 for every $x\in X$ whose reduced word begins with the inverse $s^{-1}$ of a generator in $S$, and likewise a generator $a_{s,x} = 1$ for every $\overline{sx}\in X$ whose reduced word begins with $s$ for some $s\in S$ (note that for fixed $s$ the element
 $\overline{sx}$ runs over $X$ as $x$ does).  In other words, if $S$ is finite, say of order $m$, and $X$ is finite, say of order $j$, then exactly $j-1$ elements from our original list of generators $a_{s,x}$ are equal to 1, so {\sl a subgroup of index $j$ in a free group with $m$ generators is free on
 $jm + 1 - j$ generators}.  This remarkable result says that certain subgroups of finitely generated groups have to be finitely generated, but might require more generators than the group itself.
