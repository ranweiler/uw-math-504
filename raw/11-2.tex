Continuing from last time, we have learned that Ext$_R^i(--,N)$ is a contravariant functor from left $R$-modules to abelian groups, while Ext$_R^i(M,--)$ is a covariant functor from $R$-modules to abelian groups.  Attached to any short exact sequence
$0\rightarrow A\rightarrow B\rightarrow C\rightarrow 0$ of left $R$-modules and another $R$-module $M$, it turns out that we now get a {\sl long} exact sequence
$0\rightarrow \hom_R(C,M)\rightarrow\hom_R(B,M\rightarrow\hom_R(A,M)\rightarrow\,$
Ext$_R^1(C,M)\rightarrow\,$Ext$_R^1(B,M)\rightarrow\,$Ext$_R^1(A,M)\rightarrow\,
Ext$_R^2((C,M)\rightarrow\,$Ext$_R^2(B,M)\rightarrow\cdots$
which fills out the subshort exact sequence of its first four terms that we saw earlier with Ext groups.  We have a similar long exact sequence obtained by reversing the order of $A,B,C$ and inserting the $M$ as the first argument in the $\hom$ and Ext groups.  Finally, starting with the projective resolution $\{P_i\}$ of $M$ we can tensor with a fixed {\sl right} $R$-module $N$ to obtain a chain complex (just like a cochain complex, but this time ending rather than starting with 0) $\{P_i\otimes_R N\}$ of abelian groups, whose homology (so-called rather than cohomology, this being a chain complex rather than a cochain complex) groups are called Tor groups and denoted Tor$^R_i(M,N)$.  These are again independent of the choice of projective resolution of $M$ and the functors Tor$^R_i(M,--)$ and
Tor$^R_i(--,N)$ are both covariant (from $R$-modules to abelian groups).  As $\otimes_R$ is now right but not left exact, the version of the long exact sequence attached to the short exact sequence
$0\rightarrow A\rightarrow B\rightarrow C\rightarrow 0$ is now
$\cdots\,$Tor$^R_1(M,A)\rightarrow\,$Tor$^R_1(M,B)\rightarrow\,$Tor$^R_1(M,C)\rightarrow
M\otimes_R A\rightarrow M\otimes B\rightarrow M\otimes C\rightarrow 0$
where $A,B,C$ are left $R$-modules and $M$ is a right $R$-module.  I hope to prove the existence and develop more properties of these long exact sequences later; for now I move on to (I hope) easier material.

Let $G$ be a group.  I will be discussing $G$-modules (vector spaces $V$ over a field $K$ equipped with a {\sl linear} $G$-action, so that $g\cdot (v_1 + v_2) = g\cdot v_1 + g\cdot v_2$ and
$g\cdot (kv) = kg\cdot v$, for all $g\in G, v_1,v_2,v\in V, k\in K$.  I usually write $gv$ instead of
$g\dot v$.  When I discussed group actions on finite sets earlier, I observed that an action of $G$ on a finite set $S$ is equivalent to a homomorphism from $G$ to the group Perm$(S)$ of all permutations of $S$; in a similar manner, given a $G$-module $V$ we get a homomorphism $\pi$ from $G$ to the general linear group $GL(V)$ of all $1-1$ linear maps from $V$ onto itself.  I will assume henceforth that $V$ is finite-dimensional over $K$ and that $G$ is finite, though I may look at a few examples where $G$ is infinite later.  Either the vector space $V$ or the homomorphism $\pi$ is often called a
{\sl representation} of $G$ (as it represents the abstract elements of $G$ by concrete square matrices).  Let's look at a couple of examples.  If $G$ is the group of quaternion units, let $H$ be the quaternions (real linear combinations of $1,i,j,k$, where $\pm i,j,k$ multiply in the same way as for the quaternion units).  Then $H$ ic a vector space over the complex numbers $\Bbb C$, where the complex scalars act on $H$ by {\sl right} multiplication.  Then $G$ acts on $H$ by left multiplication.  Fixing the basis $i,j$ of $H$, we find that the matrices $I,J$ by which $i,j$ act on $H$ are given by
$\pmatrix i&0// 0& -i// \endpmatrix, \pmatrix 0&-1\\ 1&0\\ \endpmatrix$, respectively.  Another way to make $G$ act linearly, this time on $\Bbb C$, is to decree that $\pm 1,\pm i$ act trivially, while $\pm j,\pm k$ act by $-1$.

 A $G$-homomorphism between two $G$-modules $V,W$ is a $K$-linear map $\pi$ from $V$ to $W$ such that $\pi g(v) = g\pi(v)$ (i.e. $\pi$ commutes with the action of $G$).   If $\pi$ is an isomorphism (in the usual sense of being $1-1$ and onto) then we call the modules (or representations) $V$ and $W$ {\sl equivalent}.  The $G$-module $V$ is called {\sl simple} or {\sl irreducible} if its only submodules (in the obvious sense) are 0 and $V$.  The key result in the study of simple $R$-modules, where $R$ is a ring, is Schur's Lemma (which you will prove in homework this week); in order to apply it in our setting, we need to realize our $G$-modules $V$ as modules over a suitable ring.  To that end, we from the {\sl group algebra} $KG$, consisting by definition of all finite formal sums $\sum_{g\in G} k_g g$, where the $k_g$ lie in $G$.  It is clear (by linearity) what
$(\sum_g k_g g)v$ should be, for any $v\in V,k_g\in K$, namely $\sum_g k_g gv$, making this definition, we realize $V$ as a left $KG$-module as desired.  (Note that $KG$ is not commutative as a ring unless $G$ is abelian as a group.)  Clearly the $KG$-submodules of $V$ are the same as the $G$-submodules, so $V$ is irreducible over $G$ if and only if it is so over $KG$.  The nicest behavior occurs when $K=\Bbb C$, the complex numbers, or more generally any algebraically closed field of characteristic 0.  Here for example if $G$ is cyclic of order $n$, then we get a family of irreducible one-dimensional representations of $G$ by decreeing that a fixed generator $g$ of $G$ act by the complex scalar $e^{2\pi ik/n}$, where $k$ lies between 0 and $n-1$.  This family turns out to account for all the irreducible representations of $G$; note that these would not be available to us if we worked over $\Bbb R$, as crucial $n$-th roots of 1 would be missing.