Let $G$ be a finite group.  We have now seen that $R=\Bbb CG$, the complex group algebra of $G$, is isomorphic as a ring to a direct sum of matrix rings $M_{n_i}(\Bbb C)$ over $\Bbb C$.  Up to isomorphism, every irreducible left $R$-module takes the form $\Bbb C^{n_i}$ for some $i$, where the summand $M_{n_i}(\Bbb C)$ of $R$ acts on $\Bbb C^{n_i}$ by left multiplication, while the other matrix ring summands act by 0.  We call $R$ the {\sl regular representation of $G$} and we observe (as in the last lecture) that each irreducible $G$-module $M=\Bbb C^{n_i}$ appears exactly
$\dim M = n_i$ times in $R$.  Now we want to compute how many nonisomorphic irreducible modules $R$ has.  We do so in a rather sneaky way, by computing in two different ways the dimension of the center $C$ of $R$ over $\Bbb C$.  On the one hand, any element of $C$ acts (as an $R$-bimodule homomorphism) by a complex scalar on the minimal two-sided ideals of $R$, which are irreducible as $R$-bimodules, so $C$ is a direct sum of copies of $\Bbb C$, one for each ideal summand of $R$, or equivalently one for every distinct irreducible representation of $G$.  On the other hand, a direct examination of a typical element $x=sum_g k_g g$ of $R$ shows that it lies in $C$ if and only if
$hx = xh$ for all $h\in G$, or if and only if $hxh^{-1} = x$ for all $h\in G$, or if and only if
$k_g = k_{g'}$ whenever the group elements $g,g'$ are conjugate in $G$.  Hence the dimension of $C$ over $\Bbb C$ equals the number of conjugacy classes in $G$, and this is the number of distinct irreducible $G$-modules.

In particular, if $G=A$ is abelian, say of order $n$, then $A$ has $n$ conjugacy classes, each consisting of a single element, and accordingly $n$ distinct irreducible representations, each necessarily of dimension 1, since the sum of the squares of their dimensions must be $n$, the order of $A$.  Once we know the irreducible representations of $A$ all have dimension 1, it is easy to find all of them.  Write $A$ as the direct product of cyclic groups, say of orders $n_1,\ldots,n_m$, with respective generators $g_1,\ldots,g_m$.  Then each $g_i$ must act by a complex $n_i$th root of $1$, say $r_i$, on any 1-dimensional module; we have $n_i$ choices for $r_i$ and accordingly
$\prod n_i = n$ choices for an irreducible representation of $A$; so we have all of them.  As another example, if $G=S_3$, the symmetric group on three letters, then we know that $G$ has a two-dimensional irreducible representation (realized by looking at the action of $G$ on an equilateral triangle centered at $(0,0)$ in $\Bbb C^2$ via symmetries of this triangle.  This representation is irreducible as there is no line in $\Bbb C^2$ that it preserves.  Besides this representation we have the trivial one on $\Bbb C$, where all elements of $G$ act trivially, and the sign representation on
$\Bbb C$, where even permutations act by 1, odd ones by $-1$.  The sum of the squares of the dimensions of these representations is $4+1+1= 6$, so again we have all of the irreducible representations.  There are three of them, matching the number of conjugacy classes in $G$.

Now an obvious question is how to construct irreducible representations of $G$, particularly if $G$ is large.  This is difficult to do directly, attaching a possibly large matrix $\pi(g)$ to every element of $G$, but fortunately the essential information in $\pi(g)$ can be distilled down to a single number, namely its trace (the sum of its diagonal entries, or of its eigenvalues).  One might wonder why we do not use the determinant $\det(\pi(g))$ instead, as a more famous number attached to a matrix, but it turns out for most $\pi$ that $\det(\pi(g)) = 1$ for all $g\in G$, so that it does not give useful information about
$\pi$.  Accordingly, we call the trace tr~$\pi(g)$ the {\sl character} of $\pi$, regarded as a complex-valued function on $G$, and denote it by $\chi_\pi$.  Since the matrices $\pi(g)$ have finite order in $GL_n(\Bbb C)$ for some $n$, each is diagonalizable with eigenvalues all $m$th roots of 1 in $\Bbb C$ for some $m$ (in fact we may take $m= | G |$, the order of $G$).  A consequence is that $\chi_\pi(g^{-1} = \overline{\chi_\pi(g)}$, the complex conjugate of $\chi_\pi(g)$, since the eigenvalues of $\pi(g^{-1})$ are the inverses of the eigenvalues of $\pi(g)$, and each such eigenvalue lies on the unit circle in $\Bbb C$.  (In fact, since the map $g\rightarrow\pi(g^{-1})^t$ is a homomorphism of $G$ into $GL_n(\Bbb C)$ whenever $\pi$ is, where the superscript $t$ denotes transpose, we see that
$\overline{\chi_{\pi}$ is the character of a representation whenever $\chi_\pi$ is.)  Furthermore, since the trace of two similar matrices is always the same, the function $\chi_\pi$ is constant on conjugacy classes of $G$.  The number of such classes, as we saw above, equals the number of distinct irreducible representations of $G$; we will see later that the characters of $\chi_\pi$ of the distinct irreducible representations of $G$ form a basis for the space of all complex-valued functions on $G$ that are constant on conjugacy classes.  

Let $V,V'$ be distinct irreducible representations of $G$, of respective dimensions say $n$ and $m$, and let $\pi,\pi'$ be the corresponding homomorphisms from $G$ to $GL_n,(\Bbb C),
GL_m(\Bbb C)$.  Then the only $G$-homomorphism from $V$ to $V'$ is the 0 map; but the other hand the proof of Maschke's Theorem shows that if $\mu$ is any $\Bbb C$-linear map from $V$ to $V'$, then $(1/ | G |)\sum_{g\in G} \pi'(g)\mu\pi(g^{-1})$ is a $G$-homomorphism $f$ and so must be 0.  Taking $\mu$ to have matrix having 1 as its $ij$-entry and 0s everywhere else, for indices $i,j$ with
$1\le i\le m,1\le j\le m$, we deduce that $\sum_g \pi(g)_{jj} \overline{\pi'(g)}_{ii} = 0$ (by looking at the $ij$-entry of the matrix of $f$), where $\pi(g)_{ij}$ is the $ij$-entry of the matrix $\pi(g)$; summing over all indices $i,j$, we deduce that $\sum_g \chi_\pi(g) \overline{\chi_{\pi'}(g}} = 0$.  We will deduce further relations among the characters $\chi_\pi$ next time.
