\input amstex
\input amssym.def
\loadmsbm
\nopagenumbers
\magnification=\magstep1
\centerline{\bf HOMEWORK \#5, DUE 11/5}
\bigskip
\centerline{\bf MATH 504A}
\bigskip
\noindent 1. (a) Let $R$ be an integral domain with field of fractions $K$.  Show that any vector space over $K$ is injective as an $R$-module.

(b) Let $R=k[x,y]$, the ring of polynomials in two variables $x,y$ over a field $k, K$ the field of fractions of $R$.  Show that the quotient $K'=K/Rx$ of $K$ by the ideal generated by $x$ in $R$ is not injective as an $R$-module, by exhibiting a nonprincipal ideal $I$ of $R$ and an $R$-module map from $I$ to $K'$ that does not extend to $R$.
\vskip .5in
\noindent 2. In the remaining problems $R$ is a non necessarily commutative ring (always with 1).  Using Zorn's Lemma, show that $R$ has a maximal (proper) two-sided ideal and a maximal left ideal.
\vskip .5in
\noindent 3. Let $S$ be a simple left $R$-module (so that $S$ has no submodules apart from itself and 0).  Show that the ring Hom$_R(S,S)$ of $R$-homomorphisms from $S$ to itself is a division ring (obeying all axioms of a field except commutativity of multiplication).
\vskip .5in
\noindent 4. Assume now that every left $R$-module is projective.   Use Zorn's Lemma to show that $R$ is the direct sum of simple two-sided ideals $R_i$, each generated as a left $R$-module by a single element $e_i$, and that there are only finitely many such ideals $R_1,\ldots,R_n$.  Show also that we may choose the $e_i$ so that $\sum_i e_i = 1, e_i^2 = e_i, e_i e_j = 0$ if $i\ne j$.
\vskip .5in
\noindent 5. Continuing with the setting of the last problem, show that each $R_i$ is in turn the sum of finitely many simple left ideals of $R_i$ and that any two such ideals are isomorphic as left modules.
\end