Continuing where we left off last time, let $M,N$ be left $R$-modules.  Last time we constructed a projective resolution of $M$:  $\cdotsP_n\cdots\rightarrow P_0 \rightarrow M\rightarrow 0$; applying
$\hom(--,N)$ and omitting the first term, we have a sequence
$0\rightarrow\hom(P_0,N)\rightarrow\hom(P_1,N)\rightarrow\cdots$
which is such that if we let $d_i$ denote the map from $\hom(P_i,N)$ to $\hom(P_{i+1},N)$, then
$d_{i+1} d_i = 0$.  We call such a sequence $C$ a {\sl cochain complex}.  We may form the quotient
 $K_i/I_{i-1}{$ of the kernel $K_i$ of $d_i$ by the image $I_{i-1}$ of $d_{i-1}$; the elements of $K_i$ are called {\sl $i$-cocycles} and those of $I_{i-1}$ {\sl $(i-1)$-coboundaries}.   The quotient
 $K_i/I{i-1}$ is then called the $i$th {\sl cohomology group $H^i(C)$ of $C$}; in this particular setting it is denoted Ext$_R^i(M,N)$ and called the $i$th Ext group of $M$ and $N$ (as $R$-modules).  (This group is only an abelian group, not an $R$-module.)  Here \lq\lq Ext" should be thought of as standing for \lq\lq extension"; it turns out that Ext$_R^1(M,N)$ measures extensions of $M$ by $N$, that is, all short exact sequences $0\rightarrow N\rightarrow P\rightarrow M\rightarrow 0$ of $R$-modules (up to an equivalence defined later).  We call the functors Ext$_R^i(--,N)$( {\sl higher (right) derived functors of $\hom$}, since the functor $\hom(--,N)$ is left but not right exact; we will later see that a short exact sequence of $R$-modules gives rise to a long exact sequence whose first three nonzero terms are
 $\hom$ groups and whose remaining terms are Ext groups.  It also turns out that the groups Ext$_R^i(M,N)$ do not depend on our choice of projective resolution of $M$.  We now give some examples.

Suppose first that $R = \Bbb Z, M = \Bbb Z_n$.  Then a projective resolution of $M$ is given by
$0\rightrrow\cdots\rightarrow 0\rightarrow\Bbb Z\rightarrow\Bbb Z\rightarrow\Bbb Z_n\rightarrow 0$, where the map from $\Bbb Z$ to $\Bbb Z$ is multiplication by $n$; in effect this is a finite resolution.  Taking Ext groups, we find that Ext$_{\Bbb Z}^0(M,N)$ consists of all homomorphisms from $\Bbb Z$ to $N$ vanishing on multiples of $n$, or equivalently homomorphisms from $\Bbb Z_n$ to $N$, while
Ext$_{\Bbb Z}^1(M,N)$ is $\hom(\Bbb Z,N)$ modulo $\hom(n\Bbb Z,N)$, which is isomorphic to $N/nN$.  The higher Ext groups Ext$_{\Bbb Z}^i(M,N)$ are 0 (for $i\ge2$).   For a more interesting, but more complicated example, take $R = \Bbb Z_n, M = \Bbb Z_d$, where $d$ is a divisor of $n$, say
$n=dm$.  Now a projective resolution of $M$ is given by
$\cdots\Bbb Z_n\rightarrow\Bbb Z_n\rightarrow\cdots\rightarrow Z_n\rightarrow Z_d\rightarrow 0$, where the maps from one copy of $\Bbb Z_n$ to the next are alternately given by multiplying by $m$ and multiplying by $d$, the rightmost map from $\Bbb Z_n$ to $\Bbb Z_n$ is multiplication by $d$, and the map from $\Bbb Z_n$ to $\Bbb Z_d$is the canonical one (thought of as multiplication by $m$).  Here we have Ext$_{\Bbb Z_n}^0(\Bbb Z_d,N) = \hom(\Bbb Z_d,N)$ (as before; this is a general fact holding for left modules over any ring $R$), but now the other Ext groups toggle:
Ext$_{\Bbb Z_n}^i(\Bbb Z_d,N) = {}_m M/dM$ if $i$ is odd, where ${}_m M$ denotes
${x\in M: mx = 0\}$, while Ext$_{\Bbb Z_n}^i(\Bbb Z_d,N) = {}_d M/mM$ if $i$ is even.  Many Ext groups exhibit this periodic behavior; many others vanish in high degrees, as we saw in the first example.  We also see from these two examples that Ext$_R(M,N)$ depends on $R$ as well as
$M$ and $N$, as indicated by the notation.

More generally, as noted above, Ext$_R^0(M,N)\cong\hom_R(M,N)$ for any $R$-modules $M,N$ and any ring $R$.  If $R$ is a PID and $M$ is finitely generated, then we have seen that $M$ is the quotient of $R^n$ for some $n$ be a free submodule which is the column span of an $n\times n$ matrix $A$ over $R$.  If we make $A$ act on $N^n$ via left multiplication (regarding $N^n$ as consisting of column vectors over $N$), then Ext$_R^1(M,N)\cong n^n/AN^n$, while the Ext groups Ext$_R^i(M,N)$ are 0 for $i\ge2$.  

Why don't the Ext groups depend on the choice of projective resolution of $M$?  To answer this, we begin by noting that, given an $R$-module map $f$ from $M$ to $N$ and projective resolutions$\{P_i\}$ and $\{Q_i}\}$ of $M,N$, respectively, an easy inductive argument yields maps
$f_i:P_i\rightarrow Q_i$ making the obvious diagram combining the two resolutions commute; applying $\hom_R(--,P)$ to this diagram, we get a map from Ext$_R^i(M,P)$ to Ext$_R^i(N,P)$ (so that Ext$_R^i(--,P)$ is indeed a covariant functor).  Using something you will construct in homework called a {sl cochain homotopy}, you will show that the induced map on Ext groups is always 0 if $f=0$; whence it will follow that any two projective resolutions of $M$ give rise to isomorphic Ext groups with a fixed module $N$.