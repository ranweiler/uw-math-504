We saw last time that any finitely generated module $M$ over a PID $R$ is a direct sum of quotients of $R/(d_i)$ of $R$, where we can arrange either that $d_1 | d_2 | \cdots d_n$ or that every $d_i$ is either 0 or a power $p^i$ of an irreducible element $p$ of $R$.  We now want to see that this decomposition is unique up to reordering the factors (and multiplying the powers $p^i$ by units, which clearly does not affect $R/(p^i)$.  First look at the {\sl torsion} submodule $T$ of $M$, consisting by definition of all $m\in M$ with $rm = 0 $ for some $r\ne0\in R$.  Since $R$ is a domain this really is a submodule; passing to the quotient $M/T$ we clearly get the sum of all the copies of $R$ itself in the decomposition of $M$ as a sum of quotients of $R$.  Any two such decompositions must involve the same number of copies of $R$ (since the rank of a free module over $R$ is well defined); we call this number the {\sl free rank}of $M$.  Now let $p$ be an irreducible element of $R, m$ a positive integer, and look at the quotient $p^m T/p^{m+1}T$ (where $p^m T$ denotes $\{p^m t: t\in T\}$, and similarly for $p^{m+1} T$.  Given a quotient $N = R/(q^r)$ of $R$ where $q\in R$ is irreducible, one checks that 
$N'=p^n N/p^{n+1} N = 0$ unless $q = pu$ for some unit $u$ in $R$ and $r\ge n$; if $N'\ne0$, then
$N'\cong R/(p)$, a vector space of dimension one over the field $R/(p)$.  Hence, for any fixed $p$ and $m$, any two primary decompositions of $M$ must involve the same number of summands isomorphic to $R/(p^k)$ for some $k\ge m$, whence any two such decompositions must involve the same number of summands isomorphic to $R/(p^m)$ itself.  The upshot is that {\sl the primary decomposition of a finitely generated $R$-module $M$ is unique up to reordering the quotients of $R$ that are the summands}.  This is the uniqueness part of the classification.

The two most important applications occur when $R=\Bbb Z$ or $R=K[x], K$ a field.  In the first case we learn that {\sl any finitely generated abelian group is a finite direct product of cyclic groups, each either infinite or of prime-power order}.  A finite abelian group i$A$ s a finite direct product of cyclic groups, each of prime-power order.  If $A$ has at most $m$ solutions to the equation $x^m = 1$ for all positive integers $m$, then it must be cyclic:  no two factors can occur whose orders are powers of the same prime $p$, lest there be too many solutions to $x^{p^k}= 1$ for some $k$, so the cyclic factors have relatively prime orders and their product is again cyclic.  Since there are always at most $m$ solutions to $x^m = 1$ in any field $K$, it follows that {\sl any finite subgroup of the multiplicative group $K*$ of a field is cyclic}.  In particular, $\Bbb Z_p*$ is always cyclic, as mentioned in class last week.  Next, as in class, take $R=K[x]$ and let $V$ be a finite-dimensional vector space over $K$ equipped with a linear transformation from $V$ to $V$.  We make $V$ into a $K[x]$-module by decreeing that $x$ act on $V$ by $T:  q(x) v = q(T) v\in V$ for all $v\in V, q\in K[x]$.  Then $V$ is isomorphic to the direct sum of quotients $K[x]/(p_i^{r_i}$ for various irreducible polynomials $p_i\in K[x]$ (the quotients must all be proper since $V$ is finite-dimensional over $K$).  Given a single quotient $V'=K[x]/(q), q(x) = p^r(x) = x^n + \sum_{i=0}^{n-1} a_i x^i$, the matrix of the transformation $T$ with respect to the fairly obvious basis $1,x\ldots,x^{n-1}$ of $V'$ has ones below the main diagonal, , last column $-a_0,\ldots,-a_{n-1}$ and zeroes elsewhere.  We call this matrix the {\sl companion matrix} $C(q)$ of $q$; it has minimal and characteristic polynomials both equal to $q$ (up to sign).  In general, the matrix of $T$ with respect to a suitable basis of $V$ will be block diagonal with blocks equal to the companion matrices attached to various powers of monic irreducible polynomials over $K$; this form is unique apart from reordering the blocks.  If the characteristic polynomial of $T$ happens to have all roots in $K$ (which it always does if for example $K$ is algebraically closed, so that all polynomials over it have a full complement of roots in it), then the powers of polynomials arising in this way all take the form $(x-a_i)^{n_i}$.  In this case one generally chooses a different basis for each block, namely the powers $(x-a_i)^{n_i-1},\ldots,x-a_i,1$ and the corresponding matrix has $a_i$'s on the diagonal, ones above it, and zeroes everywhere else.  Such a matrix is called a {\sl Jordan block} and the $\sl Jordan canonical form} of a square matrix realizes it as similar to a block diagonal matrix with the blocks all Jordan blocks (possibly with different eigenvalues).  In particular, {\sl there are only finitely many similarity classes of $n\times n$ matrices having just one eigenvalue $a$}, for any such matrix $M$ is similar to one in Jordan form with $a$'s on the diagonal and so is determined by its size; the sizes involved are a set of positive integers adding to $n$.  Such unordered sets of positive integers summing to $n$ are called {\sl partitions} of $n$ and have a rich mathematical theory; they have been studied for more than two and a half centuries.