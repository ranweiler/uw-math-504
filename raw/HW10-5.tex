\centerline{Homework \#1, due 10/5}

1. Let $G$ be a finite group of order $p^m n$, where $p$ is prime and does not divide $n$.

   (a) Using a well-known formula for the number $N$ of {\sl subsets} of $G$ of order $p^m$ as a ratio of products, work out (by pairing up terms in these products) the remainder when $N$ is divided by
  $p$, showing in particular that this remainder is not 0.
  
  (b) $G$ acts on the set $\mathcal S$ of subsets $S$ of $G$ of order $p^m$by left multiplication:  
  $g\cdot S = \{gs:  s\in S}$.  Show that any such set $S$ is the union of right cosets of its stabilizer $H$ in $G$, deducing that this stabilizer has order $p^k$ for some $k\le m$ and the order of the $G$-orbit of $S$ is not divisible by $p$ if and only if this stabilizer has order $p^m$.
  
   (c) Deduce that that $G$ has at least one subgroup of order $p^m$; this is called a $p$-Sylow subgroup of $G$; in fact, show that the number $n_p$ of such subgroups is congruent to 1 mod $p$.
   
 2.  Now let $P$ be a $p$-Sylow subgroup of $G$ and $Q$ another subgroup of order $p^k$ for some $k$.  By looking at the action of $Q$ on the left cosets of $P$, show that $Q$ lies in some conjugate $gPg^{-1}$ of $P$ (for some $g\in G$).  In particular, any two $p$-Sylow subgroups of $G$ are conjugate.  Finally, use these results to show that the number $n_p$ of $p$-Sylow subgroups of $G$ divides $n$, in addition to being congruent to 1 mod $p$.
 
3.  Show that every group of order 56 has a normal subgroup, by showing that either the 2-Sylow or the 7-Sylow subgroup of such a group must be (unique and) normal.

4.  Let $q=p^k$ be a power of a prime $p$.  Assume that there exists a field $F_q$ with $q$ elements (we will see later that this is always the case).  Compute the order of the group $G=GL(n,F_q)$ of invertible n-by-n matrices over $F_q$, by counting how many choices there are for the first row of such a matrix, then counting how many choices there are for the second row, given the first one, and so on.  

5. Now you know the largest power $p^N$ dividing the order of $G$.  By looking at triangular matrices in $G$, find a $p$-Sylow subgroup of it.

