\documentclass[10pt]{article} \usepackage{amsmath, amssymb}

\title{Math 504, Modern Algebra} \date{2016/10/03}

\begin{document}

\maketitle

Last time we stated the Sylow theorems, which you will prove in homework
for this week; now we apply them. Let $p,q$ be distinct primes with
$p<q$ and $G$ a group of order $pq$. Then the numbers $n_p,n_q$ of
$p$-Sylow and $q$-Sylow subgroups of $G$ are congruent to 1 mod $p,q$,
respectively, and divide $q,p$, respectively. It follows at once that
$n_q = 1$: the only divisor of $p$ that is congruent to 1 mod $q$ is 1,
whence the $q$-Sylow subgroup is unique and normal (since any conjugate
of it is also a $q$-Sylow subgroup). The same is true of the $p$-Sylow
subgroup if $q$ is not congruent to 1 mod $p$. In this case then the
$p$- and $q$-Sylow subgroups are both normal and have trivial
intersection, whence by counting their product is all of $G$. By a
standard result in undergraduate group theory, $G$ must be the direct
product of these subgroups, which are both cyclic, whence in fact $G$
itself is cyclic, of order $pq$. This actually holds if the $p$-Sylow
subgroup is normal, even if $q$ is congruent to 1 mod $p$. Now suppose
that $q$ is congruent to 1 mod $p$ and a $p$-Sylow subgroup is {\sl not}
normal. Letting $Q$ be the $q$-Sylow subgroup and $P$ be any $p$-Sylow
subgroup, we again have that $Q$ and $P$ intersect trivially and $G =
QP$, but now the elements of $P$ need not commute with those of $Q$. We
still get an {\sl action} of $P$ on $Q$ by group automorphisms, via
conjugation: if $x\in P,y\in Q$, then $xyx^{-1}\in Q$, and if $x$ is
fixed the map sending any $y$ in $Q$ to $xyx^{-1}$ is an automorphism of
$Q$. In this situation we say that $G$ is the {\sl semidirect} product
of $P$ and $Q$, notated either as $P\ltimes Q$ or $P\rtimes Q$. More
generally, let $N,H$ be any two groups such that $H$ acts on $N$ by
group automorphisms; if $h\in H,n\in N$, denote the action of $h$ on $n$
by $h\cdot n$. Then we can make the Cartesian product $N\times H$ into a
group via $(n_1,h_1) (n_2,h_2) = (n_1 h_1\cdot n_2),h_1 h_2)$. One
checks immediately that the group axioms are satisfied; here the
multiplication is the same as for the direct product of $N$ and $H$ in
the second coordinate, but is \lq\lq twisted" in the first coordinate by
replacing $n_2$ by $h_1\cdot n_2$. We denote this group as $N\ltimes H$
or $N\rtimes H$ (or $H\rtimes N$ or $H\ltimes N$) and call it the
semidirect product of $N$ and $H$. Whenever a group $G$ has a normal
subgroup $N$ and another subgroup $H$ such that $G = NH$ and $H$ and $N$
intersect trivially, then $G$ can always be viewed as the semidirect
product of $N$ and $H$. In our current example, the full automorphism
group of the cyclic group $\mathbf{Z}_q$ of order $q$ is known to be
itself cyclic of order $q-1$ (we will learn this later) and so admits a
cyclic subgroup of order $p$ whenever $p$ divides $q-1$, which can be
used to define a nontrivial action of the cyclic group $\mathbf{Z}_p$ of
order $p$ on $\mathbf{Z}_q$ by automorphisms. The upshot is that {\sl
  any group of order $pq$ with $p,q$ distinct primes is cyclic if
  neither of $p,q$ is congruent to 1 mod the other, but a nonabelian
  group of order $pq$ exists whenever $p,q$ are distinct primes with $q$
  congruent to 1 mod $p$}.

Sometimes the Sylow theorems force one of the Sylow subgroups to be
normal, but don't tell us which one it is (because either one can be).
For example, in homework this week, you are asked to show that, given a
group $G$ of order 56, either its 2-Sylow or its 7-Sylow subgroup must
be normal; but either possibility can occur. The key idea is to argue
that if there is more than one 7-Sylow subgroup (or equivalently, no
7-Sylow subgroup is normal), then you can count how many elements of
order 7 there are in $G$, and this turns out to be so many that the
2-Sylow subgroup is forced to be normal.

We conclude our brief treatment of Sylow theory with an illustration of
how it can be used to analyze the structure of certain groups even if
they are simple (so contain no nontrivial normal subgroups). Let $G$ be
a simple group of order 60. The number $n_5$ of 5-Sylow subgroups must
then be 1 or 6, whence by simplicity it must be 6. Then $G$ acts on its
six 5-Sylow subgroups by conjugation and the action is faithful since
$G$ is simple. Since the alternating group $A_6$ is also simple, it
follows easily that $G$ is isomorphic to a subgroup of it. This subgroup
$H$ has six left cosets in $A_6$, one of which is $H$ itself; the others
are permuted by $H$. Simplicity of both $G$ and $A_5$ then forces $G$ to
be isomorphic to $A_5$: {\sl up to isomorphism, $A_5$ is the only simple
  group of order 60}.

We look at one more example, without proving it in detail. Let $G$ be a
simple group of order 168. As above we see that it must have exactly 8
7-Sylow subgroups; it acts on the set of these by conjugation. This time
however $G$ is much too small to act by all even permutations on 8
elements. To pin down its structure, start with the set $\mathbf{Z}_7$
of integers mod 7 and add a new element $\infty$ (infinity), calling the
resulting set $P=\mathbb P^1(\mathbf{Z}_7)$. For $a,b,c,d\in
\mathbf{Z}_7$ with $ad-bc=1$, define the fraction $(az + b)/(cz + d)$
for z in $P$ by decreeing it to be $a/c$ if $z=\infty$ and $c$ is not 0;
while it is $\infty$ if $c=0$ (note that then $a$ cannot be 0).
Analogously, define $(az+b)/(cz+d)$ to be $\infty$ if its numerator is 0
(since then its denominator cannot be 0). The set of all maps sending
$z\in P$ to $(az+b)/(cz+d)$ forms a group under composition, called the
group of linear fractional transformations. This group is isomorphic to
the quotient $PSL(2,\mathbf{Z}_7)$ of 2 x 2 matrices over $\mathbf{Z}_7$
of determinant 1, modulo the scalar matrices $\pm1$; the order of this
group is 168, as desired. Then it turns out (though we will not prove
this) that $G$ is necessarily isomorphic to this group.

\end{document}
