\documentclass[10pt]{article}
\usepackage{amsmath, amssymb}

\title{Math 504, Modern Algebra}
\date{2016/09/30}

\begin{document}

\maketitle

We continue with the inequality derived last time for a finite subgroup
$G$ of order $N$ of $SO_3$, having three orbits of poles, consisting of
points stabilized by $r_1,r_2,r_3$ rotations, respectively; then

\[
(1/r_1) + (1/r_2) + (1/r_3) = 1 + (2/N) > 1
\]

Assuming as we may that $r_1\le r_2\le r_3$, and recalling that the
$r_i$ are integers at least equal to 2, it is easy to see that the only
solutions for $(r_1,r_2,r_3)$ are $(2,2,M)$ [with $N=2M$], $(2,3,3)$
[with $N = 12$], $(2,3,4)$ [with $N = 24$], and $(2,3,5)$ [with $N =
  60$]. Each triple corresponds to a uniquely determined group, once the
set of poles has been specified, and in all cases the group is the full
group of symmetries (in $SO_3$) of a familiar geometric object. More
precisely, the triple $(2,2,M)$ corresponds to the dihedral group of
order $N = 2M$ of symmetries of a regular $M$-gon; the poles are the
vertices of the $M$-gon, the midpoints of its sides (rescaled so as to
have length 1), and two points at distance 1 from the plane of the
$M$-gon, one above this plane (and directly over the center of the
$M$-gon), and one below this plane. The triple $(2,3,3)$ corresponds to
the symmetry group $T$ of a regular tetrahedron; the poles are its
vertices, the midpoints of its edges, (rescaled as always to have length
1), and the centers of its faces (rescaled). The other two triples
$(2,3,4)$ and $(2,3,5)$ correspond to the respective symmetry groups
$O,I$ of a regular octahedron and icosahedron, with poles specified as
for the tetrahedron. (A cube has the same symmetry group as an inscribed
octahedron; similarly a dodecahedron has the same symmetry group as an
inscribed icosahedron.) The group $T$ is isomorphic to the alternating
group $A_4$; the orientation-preserving symmetries of a tetrahedron act
by the even permutations of its vertices. The group $O$ is isomorphic to
the symmetric group $S_4$; here the orientation-preserving symmetries
act by all permutations of the four pairs of opposite faces. Finally,
and most subtly, $I$ is isomorphic to the alternating group $A_5$: any
dodecahedron has exactly 5 inscribed cubes with vertices among its
vertices; the orientation-preserving symmetries act by all even
permutations of these 5 cubes. For more details see section 6.12 of
Artin's book Algebra, on reserve in the Math Library. We will see these
groups later, defined in a particularly striking way by generators and
relations.

Return now to a general finite group $G$. We noted in the last lecture
that an action of $G$ on a set $S$ is equivalent to a homomorphism from
$G$ to the group Perm(S) of permutations of $S$. We are particularly
interested in the case where this homomorphism is 1-1, so that $G$ is
realized as (isomorphic to) a subgroup of Perm(S). To this end, we note
that if $s$ is in $S$ and has stabilizer $H$, then the stabilizer of
another element $gs$ inits orbit is just the conjugate $gHg^{-1}$ of $H$
by the element $g$. The kernel of the homomorphism from $G$ to the
permutation group Perm(G$\cdot$s) of just this orbit is thus the
intersection of the conjugates of $H$. By studying the subgroups of a
given group $G$, we can thus decide whether $G$ is isomorphic to a
subgroup of Perm(S) for various choices of $S$. For example, if $G$ is
dihedral of order 8, then it has four cyclic subgroups generated by
reflections, forming two conjugacy classes. The intersection of the
cyclic subgroups in each conjugacy class is trivial, whence we get a
{\sl faithful} action (with trivial kernel) of $G$ on a four-element
set, namely the set of left cosets of any of these subgroups. (Of
course, we already knew that $G$ acts faithfully on the four vertices of
a square, but we just deduced this property of $G$ by looking at $G$
alone.) By contrast, if $G$ is the quaternion group of order 8,
consisting of $\pm1,\pm i,\pm j$, and $\pm k$, with $i^2 = j^2 = k^2 =
-1, ij = -ji = k, jk = -kj = i, ki = -ik = j$, then $G$ has only one
cyclic subgroup or oder 2, generated by $-1$, and indeed every
nontrivial subgroup of $G$ contains $-1$. It follows that $G$ does not
admit a faithful transitive action on any 4-element set; with a bit more
work one sees that $G$ does not admit a faithful action on any set of
size less than 8.

We conclude with a statement of the Sylow theorems for finite groups,
which you will prove next week in homework. Let $p$ be a prime number
and $G$ a finite group of order $p^m n$, where $p$ does not divide $n$.
Then the number $n_p$ of subgroups of $G$ of order $p^m$ (called
$p$-Sylow subgroups) is congruent to 1 mod $p$ and divides $n$; in
particular there is always at least one such subgroup. In addition any
two such subgroups are conjugate in $G$.

\end{document}
