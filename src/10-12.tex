\documentclass[10pt]{article}
\usepackage{amsmath, amssymb}

\begin{document}

\section*{Math 504, 10/12}

Continuing from last time,let $R$ be a PID. We will classify finitely
generated $R$-modules; we start with submodules $N$ of $R^n$. Here we
find the same situation as for $\mathbf{Z}$: {\sl any such submodule is
  isomorphic to $R^m$ for some $m\le n$}. This is also proved by
induction on $n$, in the same way as for $\mathbf{Z}$: it is clear from
the definition of PID if $n=1$; in general, given $N$, look at the set
$I$ of $a\in R$ such that $(a,a_2,\ldots,a_n)$ lies in $N$ for some
$a_i\in R$ and argue that $I$ is an ideal; if it is nonzero and
generated by $a\in R$ and $v=(a,a_2,\ldots,a_n)\in N)$, then $N$ is the
direct sum of the submodule generated by $v$ and the intersection $N\cap
0\times R^{n-1}$, with the first submodule isomorphic to $R$ because $a$
is a non-zero-divisor. Now let $M$ be any $R$-module generated by $n$
elements. Then is the quotient $R^n/N$ of $R^n$ by a submodule $N$,
which we may take to be the column space of an $n\times m$ matrix $A$
for some $m\le n$. Adding columns of zeroes if necessary, we may assume
that $m=n$. Now I claim that {\sl if we replace the column space of $A$
  by that of $PAQ$ for any $n\times n$ invertible matrices $P,Q$ over
  $R$, then the quotient of $R^n$ by the column space does not change,
  up to isomorphism}. Indeed, replacing $A$ by $AQ$ does not change its
column space, while the column space $C$ of $PA$ is such that
$R^n/N\cong R^n/C$ by the map sending the coset of any $v\in R^n$ to
that of $Pv$.

We then find matrices $P,Q$ for which the column space of $PAQ$ is
transparent. To do this, note first that any elementary column operation
on $A$, replacing one column by the sum of itself and a multiple of
another row, can be implemented by multiplying $A$ by an invertible
matrix $Q_1$ on the right; likewise any elementary row operation on $A$
can be implemented by multiplying it by an invertible matrix $P_1$ on
the left. More generally, let $x,y,z,w$ be any elements of $R$ such that
$xw - yz = 1$. Then multiplying $A$ on the right by the matrix $Q'$
having $x,y$ as the first two entires of its first column, $z,w$ as the
first two entries of its second column, with all other entries the same
as the corresponding entires in the identity matrix has the effect of
replacing the first two columns $C_1,C_2$ of $A$ by
$xC_1+yC_2,zC_1+wC_2$ and leaving all other columns unchanged. Doing
this does not change the column space. In a similar manner, if we want
to do what we just did to $C_1,C_2$ to some other pair of columns
$C_i,C_j$ instead, then we can do this by multiplying $A$ on the right
by a matrix agreeing with the identity matrix except in its $ii,ij,ji$,
and $jj$-entries. We can do the same thing to any two rows $R_i,R_j$ of
$A$ by multiplying it by a suitable matrix on the left. Now let
$a_1,a_2$ be the first two entries in the first row of $A$.and let $a$
be a gcd of these elements in $R$. There is $x,y\in R$ with $xa_1 + ya_2
= a$ and the coefficients $x,y$ have no common factor in $R$, lest a
proper multiple of $a$ divide both $a_1,a_2$, so there are $z,w\in R$
with $xw - yz = 1$. Multiplying $A$ on the right by the above matrix
$Q'$, we replace $a_1$ by $a$ and $a_2$ by a multiple of $a$. Iterating
this operation for the other rows, we can arrange to put a gcd of all
entries in each row in its first coordinate. Iterating the operation for
the new first column, we can put a gcd of all entries of $A$ into its
upper left corner, while all other entries are multiples of this one.
Finally, performing suitable row and column operations, we can zero out
the first row and column of $A$ (except for its $(1,1)$-entry). The
upshot is a new matrix $A'$ with $(1,1)$-entry $d$ such that all other
entries in its first row and column are 0 and all other entries anywhere
are multiples of $d$. Iterating this procedure, we see that {\sl we can
  replace $A$ by a diagonal matrix with diagonal entries say
  $d_1,\ldots,,d_n$, such that $d_1 | d_2 | \cdots | d_n$}. But the
quotient of $R^n$ by the column space of this last matrix is clearly the
sum of the quotients $R/(d_i)$ of $R$. Thus {\sl any finitely generated
  $R$-module $M$ is the direct sum of quotients $R/(d_i)$ of $R$, such
  that $d_1 | d_2 | \cdots | d_n$ in $R$. (Note that we allow some $d_i
  = 0$ for some $i$, but then all subsequent $d_j$ must also be 0.) This
  is called the {\sl elementary divisor decomposition of $M$} and is
  unique up to multiplying the $d_i$ by units in $R$.

There is another decomposition typically involving more but simpler
summands than this one. Recall (by the Chinese Remainder Theorem) that
anynonero proper quotient $R/(a)$ is isomorphic to the direct sum of the
quotients $R/(p_i^{m_i})$, where $p_1^{m_1}\ldots p_r^{m_r}$ is the
irreducible factorization of $a$ in $R$. Replacing each term $R/(d_i)$
in the elementary divisor decomposition as above, we get a second
decomposition of $M$ as a direct sum of quotients $R/(q_i)$ where each
$q_i$ is either 0 or a power of an irreducible element in $R$. This last
decomposition of $M$ is called the {\sl primary decomposition} and it is
unique up to reordering the summands and multiplying the $q_i$ by units.

\end{document}
