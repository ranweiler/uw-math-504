\documentclass[10pt]{article}
\usepackage{amsmath, amssymb}

\begin{document}

\section*{Math 504, 10/21}

We now look at sequences of modules linked by homomorphisms. Given a
sequence of $R$-modules $\cdots\rightarrow M\rightarrow N\rightarrow
P\rightarrow\cdots$, each one sent to the next by an $R$-module map, we
say that it is {\sl exact at $N$} if the image of the ($R$-module) map
from $M$ to $N$ equals the kernel of the map from $N$ to $P$. The
sequence is {\sl exact} if it is exact at all of its terms, except
possibly those, if any, on the extreme ends. The most important kind of
exact sequence is a {\sl short exact sequence} $0\rightarrow
N\rightarrow M\rightarrow P\rightarrow 0$; note that such a sequence is
exact if and only if the map $f$ from $N$ to $M$ is injective, the map
$g$ from $M$ to $P$ is surjective, and the image of $f$ equals the
kernel of $g$ (so that $P\cong M/N$, identifying $N$ with its image
under $f$ in $M$). Such sequences are the basic object of study of {\sl
  homological algebra}, which plays a key role in algebraic topology;
one of the aims of this course is give an introduction to this subject.
The short exact sequence $0\rightarrow N\rightarrow M\rightarrow
P\rightarrow 0$ is said to be {\sl split} if there is a submodule $P'$
of its middle term $M$ mapping isomorphically onto $P$ via the map from
$M$ to $P$, so that $M$ is the direct sum of the image of $N$ in it and
$P'$. We will be looking at important operations, called {\sl functors},
which take $R$-modules to $R$-modules; our first such operation fixes an
$R$-module $M$ and sends an arbitrary $R$-module $N$ to the set
Hom$_R(M,N)$ of $R$-module maps from $M$ to $N$; note that Hom$_R(M,N)$
is also an $R$-module in a natural way. Given a sequence $M_1\rightarrow
M_2\rightarrow M_3\rightarrow\cdots$ of $R$-modules and our fixed
$R$-module $M$, we get a sequence
Hom$_R(M,M_1)\rightarrow\,$Hom$_R(M,M_2)\rightarrow\cdots$ by
composition; accordingly, we call the functor sending each $M_i$ to
Hom$_R(M,M_i)$ (often denoted Hom$_R(M,--)$) {\sl covariant}, since the
directions of the arrows are preserved. (If they were reversed instead,
as they will be in some subsequent examples, then we call the functor
{\sl contravariant}). Our first question is whether the covariant
functor Hom$_R(M,--)$ sends one short exact sequence to another one.
Given a short exact sequence $0\rightarrow A\rightarrow B\rightarrow
C\rightarrow 0$, our functor yields the sequence
$0\rightarrow\,\,$Hom$_R(M,A)\rightarrow\,\,$Hom$_R(M,B)\rightarrow\,\,$
Hom$_R(M,C)\rightarrow0$. Here the map $f$ from
Hom$_R(M,A)\rightarrow\,\,$Hom$_R(M,B)$ is indeed injective, since $A$
embeds in $B$, and the map from Hom$_R(M,B)$ to Hom$_R(M,C)$ does indeed
have as kernel the image of Hom$_R(M,A)$ in Hom$_R(M,B)$, but the map
from Hom$_R(M,B)$ to Hom$_R(M,C)$ need {\sl not} be surjective; for
example, if $R=\mathbf{Z}, M = \mathbf{Z}_2$, and $A,B,C$ are
respectively $\mathbf{Z}, \mathbf{Z}$, and $\mathbf{Z}_2$, with the map
from $A$ to $B$ sending an integer $x$ to $2x$, then the map from
Hom$(\mathbf{Z}_2,\mathbf{Z})$ to Hom$(\mathbf{Z}_2, \mathbf{Z}_2)$ is
not surjective, as there are no nonzero homomorphisms from
$\mathbf{Z}_2$ to $\mathbf{Z}$, but there is a nonzero homomorphism from
$\mathbf{Z}_2$ to $\mathbf{Z}_2$. Thus the first four terms of our
original exact sequence $0\rightarrow A\rightarrow B\rightarrow
C\rightarrow 0$ remain exact after applying Hom$_R(M,--)$, but the full
sequence does not. We summarize this situation by saying that the
functor Hom$_R(M,--)$ is {\sl left exact}, but not {\sl right exact}. On
the other hand, there are some modules $P$ over some rings $R$ for which
this functor is {\sl exact}, preserving the full short exact sequence we
started out with. This will happen if and only if any $R$-module map $f$
from $P$ to another module $N$ will always {\sl lift} in the sense that,
given a module $M$ and a surjection $g$ from another $R$-module $M$ onto
$N$, there is a map $f'$ from $P$ to $N$ with $f= gf'$. We call such
$R$-modules {\sl projective} (over $R$); unfortunately this has nothing
to do with projective geometry or projective space (or the projective
special linear group $PSL_n$) that arise in other contexts. For example,
if $P$ is free over $R$, then the lift $f'$ always exists: let
$p_1,p_2,\ldots$ be a basis of $P$ and choose preimages $m_1,m_2,\ldots$
of the images $f(p_i)$ in $M$, so that $g(m_i) = f(p_i)$, and then take
$f'$ to be the unique $R$-module map from $P$ to $M$ sending each $p_i$
to $m_i$. More generally, if $P$ is a direct summand of a free module
$F$, so that $F\cong P\oplus Q$, then again the lift $f'$ exists: extend
$f$ to a map from $F$ to $N$ by decreeing that it be 0 on $Q$, and then
lift it as above. In fact, it turns out that {\sl the projective
  $R$-modules are exactly the direct summands of free $R$-modules}; to
see this, let $P$ be projective over $R$ and let $F$ be a free module
surjecting onto $P$; we know that such an $F$ exists. Then the identity
map from $P$ to itself must lift to $F$, so that there is a map
$f:P\rightarrow F$ such that composing $f$ with the surjection from $F$
to $P$ is the identity on $P$. This says exactly that $F$ is the direct
sum of the image $f(P)$ in it and the kernel of the surjection from $F$
to $P$, so that $P$ is (isomorphic to) a direct summand of a free
module, as claimed. Note that although the free module $R^n$ is given to
us as the direct sum of $n$ copies of $R$, there could be other ways to
write $R^n$ as a direct sum and accordingly projective $R$-modules that
are not free. The simplest example occurs when $R = \mathbf{Z}_6$: here
$R$ decomposes as the direct sum of its submodules generated by $3,2$,
which are isomorphic to $\mathbf{Z}_2,\mathbf{Z}_3$, respectively. Hence
$\mathbf{Z}_2$ and $\mathbf{Z}_3$ are projective over $\mathbf{Z}_6$
(but clearly not free).

\end{document}
