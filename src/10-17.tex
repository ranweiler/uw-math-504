\documentclass[10pt]{article}
\usepackage{amsmath, amssymb}

\begin{document}

\section*{Math 504, 10/17}

We now describe an interesting application of the rational canonical
form to a mathematical card trick (!) Begin by noting that the companion
matrix $C(p^i)$ of a power of an irreducible polynomial $p$ over a field
$K$ is the only matrix up to similarity of its size with minimal
polynomial $p^i$, since in general, as noted in the last lecture, the
minimal polynomial of any rational canonical form is the least common
multiple of the polynomials of which its blocks are the companion
matrices. In particular, the transpose $C(p^i)^t$ of $C(p^i)$ has the
same minimal polynomail $p^i$, so is similar to $C(p^i)$; in later
homework you will show that any square matrix over any field is similar
to its transpose. Now let $K=\mathbf{Z}_2$, the field with two elements,
and let $L=\mathbf{Z}_2[x]/(x^5 + x^2 + 1)$. As it is easy to check that
$x^5 + x^2 + 1$ is irreducible in $\mathbf{Z}_2[x]$, it follows that $L$
is a field that is 5-dimensional as a vector space over $K$, so $L$ has
exactly 32 elements. A basis of $L$ over $K$ is given by the powers
$1,x,\ldots,x^4$; the matrix $M$ of multiplication by $x$ as a
$K$-linear transformation of $L$ with respect to this basis is just the
companion matrix of $x^5 + x^2 + 1$. Its 31st power is the identity,
since we must have $x^31 = 1$ in $L$, but no lower power $x^i$ of $x$
has $x^i v = v$ for any nonzero $v$ in $L$. By the remark made at the
beginning, the transpose $N=M^t$ of $M$ also has $N^{31} = I$ but $N^i
v\ne v$ for any positive $i<31$ and nonzero $v$. Multiplying $M^t$ by a
column vector $(a_1,\ldots,a_5)$, we get $(a_2,\ldots,a_5,a_1 + a_3)$.
It follows that if we define a 31-element sequence $b_1,\ldots,b_{31}$
of elements of $K$ recursively via $b_1=b_2=b_3=b_4=0, b_5=1, b_n =
b_{n-3}+b_{n-5}$ for $n\ge6$, then the *consecutive* 5-bit subsequences
$b_i,\ldots,b_{i+5}$ run through all the nonzero 5-bit sequences of
elements of $K$, each occurring once and only once; here $1\le i\le 31$
and the addition in the subscripts takes place modulo 31. If we
continued the sequence $b_1,b_2,\ldots$ by the same recursive rule
inductively, we would get its first block $b_1,\ldots,b_{31}$ repeated
indefinitely. Instead we continue for just one more term, setting
$b_{32}= 0$. Now the consecutive 5-bit subsequences $b_i,\ldots,b_{i+5}$
of this sequence $b_1,\ldots,b_{32}$ run through *all* the 5-bit
sequences of elements in $K$, each occurring exactly once, where the
addition of subscripts now takes place modulo 32. Such a 32-bit sequence
is called a {\sl deBruijn sequence} (of length 32) and is of interest in
combinatorics; it can be constructed in several ways quite different
from the above. We use this sequence to perform a card trick, as
follows. Discard the nines through the kings from an ordinary 52-card
deck of cards, leaving just 32 cards. Encode each card in the deck by a
5-bit sequence, using the first three bits to encode the rank
(ace=000,...,eight=(111)), while the last two bits encode the suit
(spades=00, hearts=(01), diamonds=(10), clubs=(11). Now arrange the
32-card deck so that the $i$th card from the top is the one encoded by
the $i$th 5-bit subsequence. Thus the first card, encoded by 00001, is
the ace of spades; the next one, encoded by 00010, is the ace of hearts,
and so on. The resulting order of cards will seem completely random to
anyone who examines the deck, but you can essentially memorize it,
working out in your head what the card following any given one is, by
applying the recurrence $b_n = b_{n-3} + b_{n-5}$. Thus you can use this
deck to \lq\lq prove" that you have ESP, announcing in advance what the
order of the cards in it will be. This trick can be adapted to any deck
(not necessarily one of ordinary playing cards) whose size is any power
$2^k$ of 2, provided only that one can encode the cards by $k$-bit
sequences; one uses the existence of a field of order $2^k$ (which we
will prove later) to construct the required deBruijn sequence.

We now consider more general modules over any commutative ring $R$. Let
$M,N$ be two $R$-modules. A {\sl bilinear} map from $M\times N$ to
another $R$-module $P$ is a map $f$ such that $f(m_1+m_2,n) =
f(m_1,n)+f(m_2,n), f(m,n_1+n_2) = f(m,n_1)+f(m,n_2), f(rm,n) = f(m,rn) =
rf(m,n)$ for all $m,m_1,m_2\in M, n,n_1,n_2\in N, r\in R$. We construct
a new module $M\otimes N$ or $M\otimes_R N$ called their {\sl tensor
  product} such that bilinear maps from $M\times N$ to $P$ are in
natural bijection to $R$-linear maps from $M\otimes N$ to $P$, as
follows. Start with the free $R$-module on the Cartesian product
$M\times N$ and mod out by the submodule $S$ generated by $(m,n) +
(m',n) - (m+m',n), (m,n) + (m,n') - (m,n+n'), (rm,n) - (m,rn),(m,rn) -
r(m,n)$ for all $r\in R,m,m'\in M, n,n'\in N$; note that any bilinear
map would have to send any generator of $S$ to 0. The individual
elements $m\otimes n$ of $M\otimes N$ are called {\sl tensors}; note
that a general element of $M\otimes N$ is a sum of tensors rather than a
single tensor. As a first example, suppose that $M$ and $N$ are both
free over $R$, say ranks $m,n$, respectively, with respective bases
$x_1,\ldots,x_m$ and $y_1,\ldots,y_n$. Then it is not difficult to show
that a bilinear map $f$ from $M\times N$ to $P$ is completely determined
by the images $v_{ij}=f(x_i,y_j)$ of ordered pairs of basis vectors,
which can be any elements of $P$. It follows that $M\otimes N$ is also
free over $R$, with basis $\{x_ i\otimes y_j\}$. In general, however,
tensor products can behave in quite unexpected ways; we will explore
some of these in the next few lectures.


\end{document}
