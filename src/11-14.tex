\documentclass[10pt]{article}
\usepackage{amsmath, amssymb}

\begin{document}

\section*{Math 504, 11/14}

We now look at a very interesting family of finite groups that deserves
to be much better known (yet does not seem to have a name in the
standard literature); I will call groups in this family Clifford groups.
The $n$th of these, call it $G_n$, is generated by $n$ elements
$a_1,\ldots,a_n$ subject to the relations $a(i^2 = \epsilon,\epsilon^2 =
1, a_i a_j = \epsilon a_j a_i$ if $i\ne j$. Thus the elements of $G_n$
are exactly the products $a_{i_1}\cdots a_{i_k}$, where $i_1 < \cdots <
i_k$, together with $\epsilon a_{i_1}\cdots a_{i_k}$, and the order of
$C_n$ is $2^{n+1}$. Its center $Z_n$ is just $\{1,\epsilon\}$ if $n$ is
even, but is $\{1,\epsilon,a_1\cdots a_n,\epsilon a_1\cdots a_n\}$ if
$n$ is odd. If $n$ is congruent to 1 mod 4, then $Z_n$ is cyclic and
generated by $a_1\cdots a_n$; if $n$ is congruent to 3 mod 4, then $Z_n$
is the product of two cyclic groups of order 2. Hence $C_n$ has $2^n +
1$ conjugacy classes if $n$ is even (two of size 1, $2^n - 1$ of size 2)
but $2^n + 2$ conjugacy classes if $n$ is odd. The easiest (but most
boring) representations of $G_n$ are those in which $\epsilon$ acts
trivially; the quotient group $G_n/<\epsilon>$ is the product of $n$
copies of $\Bbb Z_2$, so $G_n$ has $2^n$ representations of degree one
(for all $n$, both even and odd). There is only one irreducible
representation $V_n$ of $C_n$ remaining if $n$ is even, which must have
degree $2^{n/2}$ (so that the sum of the squares of the irreducible
degrees equals $2^{n+1}$, the order of $G_n$). If $n$ is odd, there are
two irreducible representations remaining, each of degree $2^{(n-1)/2}$;
to see that they must have the same degree, observe that $C_n$ is
generated by the copy of $G_{n-1}$ inside it and the central element
$a_1\cdots a_n$, so the two remaining representations are isomorphic to
$V_{n-1}$ as representations of $G_{n-1}$, with $a_1\cdots a_n$ acting
by $\pm i$ if $n$ is congruent to 1 mod 4, or $\pm 1$ if $n$ is
congruent to 3 mod 4. Now one can form the quotient of the group algebra
$\Bbb C G_n$ by the ideal generated by $\epsilon+1$; this quotient is
generated (now as an algebra over $\Bbb C$) by $a_1,\ldots,a_n$ subject
to the relations $a_i^2 = -1, a_i a_j = - a_j a_i$ if $i\ne j$. This
quotient $C_n$ is well known in the literature and is called the {\sl
  Clifford algebra} (over $\Bbb C$). It is isomorphic to a single matrix
ring $M_{2^{n/2}}(\Bbb C)$ if $n$ is even and the direct sum of two
isomorphic matrix rings $M_{2^{(n-1)/2}}$ if $n$ is odd. Its
representation(s) $V_n$ as above (one of them if $n$ is even, two if $n$
is odd) is/are called {\sl half-spin representations} and play an
important role in physics. Although $C_n$ is not a group under
multiplication, it admits a remarkable multiplicative subgroup,
generated by all linear combinations $\sum z_i a_i$ with $z_i\in\Bbb
C,\sum z_i^2 = 1$. This is called the {\sl pin group} Pin$_n(\Bbb C)$
and turns out to be a double cover of the orthogonal group $O_n(\Bbb
C)$; likewise products of evenly many combinations of the above type
form a subgroup of index 2 in Pin$_n(\Bbb C)$ denoted Spin$_n(\Bbb C)$
and called the {\sl spin group}. It is a simply connected double cover
of $SO_n(\Bbb C)$. The pin and spin groups have analogues over $\Bbb R$
as well, defined by restricting the coefficients $z_i$ as above to the
real numbers; these too are important in physics.

We now change gears slightly and investigate the entries in character
tables (and certain related complex numbers) in more detail. Call a
complex number $z$ an {\sl algebraic integer} if it satisfies a {\sl
  monic} polynomial with integral coefficients; this is a stronger than
being algebraic (over $\Bbb Q$). Equivalently, $z$ is an algebraic
integer if the subring $\Bbb Z[z]$ of $\Bbb C$ generated by $\Bbb Z$ and
$z$ is finitely generated as a $\Bbb Z$-module, so that if $z$ is an
algebraic integer, so is every element of $\Bbb Z[z]$. Now it is easy to
see that if $x$ and $y$ are algebraic integers, then the subring $\Bbb
Z[x,y]$ generated by both $x$ and $y$ is generated as a $\Bbb Z$-module
by finitely many products $x^i y^j$, whence all elements of $\Bbb
Z[x,y]$ are again algebraic integers and in particular $x+y,x-y,xy$ are.
On the other hand, {\sl a rational number $r/s$ is an algebraic integer
  if and only if it is an integer}, for if we write down a monic
polynomial over $\Bbb Z$ satisfied by $r/s$ and clear the denominators,
we find by looking at the greatest common divisor of the terms that
$s=\pm1$. Now the entries $\chi(g)$ in the character table of a finite
group $G$, being sums of roots of 1 in $\Bbb C$, must be algebraic
integers, so in particular no nonintegral rational number can occur in a
character table.

A beautiful consequence for the symmetric group $S_n$ is that {\sl the
  entries in its character table are all integers}. To see this, note
first that $S_n$ has a very special group-theoretic property: the order
of any element $g$ is the least common multiple $m$ of the lengths of
the cycles in its cycle decomposition, whence any power $g^a$ with
gcd$(a,m) = 1$ is a product of cycles of the same lengths and so is
conjugate to $g$. Now for any representation $\pi$ of $S_n$, the matrix
$\pi(g)$ will have all eigenvalues $m$-th roots of 1 in $\Bbb C$, whence
they will lie in the field $F_m = \Bbb Q[e^{2\pi i/m}]$ generated by all
these $m$-th roots, as will the trace $\chi(g)$ of $\pi(g)$. Passing
from $g$ to $g^a$ (with gcd$(a,m) = 1$), we replace every eigenvalue of
$\pi(g)$ by its $a$-th power, and yet the trace must remain the same,
since $g^a$ is conjugate to $g$. Now we will see next term that for any
such $a$ there is an automorphism of $F_m$ mapping $e^{2\pi i/m}$ to its
$a$-th power, and likewise for any power of $e^{2\pi i/m}$. The set of
all such automorphisms of $F_m$ is a group under composition, called the
Galois group of $F_m$, and th e only elements of $F_m$ fixed by every
element of this group are those of $\Bbb Q$. Hence we must have
$\chi(g)\in\Bbb Q$; but now since $\chi(g)$ must be an algebraic
integer, we must in fact have $\chi(g)\in\Bbb Z$, as claimed.

There is another very nice result proved using algebraic integers.
Recall the sums $S_g = \sum_{g\in C_g} g$ in $\Bbb CG$ introduced
earlier, where $C_g$ is the conjugacy class of $G$. We saw before that
$S_g$ is central in $\Bbb CG$, so acts as a complex scalar $z_{g,\pi}$
on any irreducible representation $\pi$ of $G$. Now we can say something
about $z_g$. Note that the product $S_g S_g'$ of any two $S_g$'s is
again central in $\Bbb CG$ and has only nonnegative integers as
coefficients of group elements. It follows that $S_g S_g'$ is a
nonnegative integral combination of $S_h$'s (as $h$ runs over $G$) and
the same holds of $z_{g,\pi} z_{g',\pi}$ and the $z_{h,\pi}$'s. This
says that the subring $\Bbb Z[z_{g,\pi}:g\in G]$ is finitely generated
as a $\Bbb Z$-module, whence every $z_{g,\pi}$ must be an algebraic
integer. Its value must be $c_g \chi(g)/\chi(1)$ where $\chi$ is the
character of $\pi$ and $c_g$ is the size of the conjugacy class $C_g$,
whence {\sl $c_g \chi(g)/\chi(1) $ is an algebraic integer}. Summing
over one element of every conjugacy class of $G$ and using the
orthonormality of the irreducible characters, we deduce that the ratio
$| G |/\chi(1)$ is an integer for every irreducible character $\chi$:
{\sl the degree of any irreducible representation divides the order of
  the group}. Thus it was no coincidence that the representations of the
Clifford group $G_n$ all have degrees that are powers of 2.

\end{document}
