\documentclass[10pt]{article}
\usepackage{amsmath, amssymb}

\begin{document}

\section*{Math 504, 9/28}

I will begin with group theory, starting with group actions on sets.
Given a group $G$ and a set $S$, an action of $G$ on $S$ is a rule that
assigns to each pair $(g,s)$ in $G x S$ an element $t$ of $S$, usually
denoted $gs$, such that $1s = s$ for all $s$ in $S$, where $i$ is the
identity element of $G$ and $h(gs) = (hg)s$ for $g,h$ in $G$, $s$ in
$S$. (More precisely, this defines a \textbf{left} action of $G$ on $S$;
had we written $sg$ for the action of the element $g$ on the element
$s$, the natural axiom would have been $(sg)h = s(gh)$; this defines a
\textbf{right} action). It is not difficult to verify that a left action
of $G$ on $S$ is equivalent to a homomorphism $\pi$ from $G$ to the
permutation group Perm($S$) of all bijections from $S$ to $S$, where we
decree that $\pi(g)(s) = gs$, where $gs$ is the action of $g$ on $s$. If
$G$ acts on $S$ and $s$ is an element of $S$, then the set of $g$ in $G$
such that $gs = s$ is an important subgroup of $G$, called the
\textbf{stabilizer} of $s$ and denoted $G^s$. Likewise, we have an
important subset $Gs$ of $S$, defined to be the set of all $gs$ as $g$
runs through $G$ and called the orbit of $s$. If $H = G^s$, then it is
easy to check that any two elements of the same left coset $gH$ of $H$
in $G$ map $s$ to the same element, and in fact two elements of $G$ map
$s$ to the same element if and only if they lie in the same left coset
of $H$ in $G$. If $G$ is finite, it follows by Lagrange's Theorem (which
I assume you have seen) that the order $|G|$ of $G$ equals the product
of the orders $|G^s||Gs|$ of the orders of the orbit and stabilizer of
any $s$ in $S$: this important formula, called the Orbit Formula, will
be used constantly in this course (and beyond). If the set $S$ has only
one $G$-orbit, then we call the $G$-action on it {\sl transitive}; in
this case we can of course replace the orbit $Gs$ in the Orbit Formula
by the entire set $S$. In general, no two orbits of $G$ in $S$ can
overlap, so another useful formula, if $S$ is finite, is that its order
equals the sum of the orders of the orbits in it.

An important but all too rarely seen example occurs when $G$ is a finite
subgroup of $SO_3$; that is, a finite group of 3 x 3 real matrices $M$
of determinant 1 such that the transpose $M^t$ of $M$ equals its inverse
$M^{-1}$, or equivalently such that $M$ preserves dot products in $R^3$:
$Mv\cdot Mw = v\cdot w$ for all vectors $v,w$ in $R^3$. Any such matrix
$M$ is such that $\det (M- I) = \det (M^t - I^t) = \det (M^{-1} - I) =
\det (M^{-1} - I)(\det M) = \det (I - M) = -\det (M - I) = 0$, whence
$M$ has an eigenvalue 1 and must fix some nonzero vector $v$ in $R^3$.
But then $M$ preserves the plane in $R^3$ perpendicular to $v$, whence
it acts by a rotation in this plane. It follows that $M$ is either the
identity or fixes exactly two unit vectors in $R^3$ (each the negative
of the other), acting by a rotation about the line through these
vectors, which also goes through the origin. Thus any $g$ in $G$ with
$g\ne1$ has exactly two poles. If $N$ is the order of $G$, there are
$2(N-1)$ poles in $R^3$ of nonidentity elements of $G$, counting each as
often as it appears as the pole of a nonidentity element. On the other
hand, if $p$ is a pole of some element of $g$ and $h$ is any other
element of $G$, then $hp$ is a pole of $hgh^{-1}$, so $G$ acts on the
set $P$ of poles of its nonidentity elements. Each pole $p_i$ in $P$
will have a stabilizer $G^p$ consisting of finitely many rotations in
$G$; if there are $r_i$ such rotations, then the number $n_i$ of
elements in the orbit of $p$ satisfies $r_i n_i = N$. Counting the
number of poles again, this time for each pole counting the number of
nonidentity elements of $G$ fixing it and observing that this number is
the same for any pole in the orbit of $p$, we get $\sum n_i(r_i - 1)$,
where there is one index $i$ for every orbit of poles in $G$. Hence

\[
2 (N - 1) = \sum_i n_i(r_i - 1)
\]

where the number of terms in the sum equals the number of orbits.
Dividing both sides by $N$, we get

\[
2(1 - 1/N) = \sum_i (1- (1/r_i))
\]

using again that $n_i r_i = N$, by the Orbit Formula. But now the left
side is less than 2, while every term in the right side is at least 1/2,
so there are at most 3 orbits. More precisely, the left side is at least
1 and each term on the right side is less than 1, so there cannot be
just one orbit. If there are two orbits, then we must have $r_1 = r_2 =
N$ (since $r_1$ and $r_2$ are less than $N$). In this case there are
exactly two poles, each lying in an orbit by itself. The group $G$ must
be a cyclic group, consisting of rotations by multiples of $2\pi/N$
about a fixed axis. If there are three orbits, then
$(1/r_1)+(1/r_2)+()1/r_3) > 1.$ We will work out the consequences of
this elementary but very famous inequality and (sketch) a classification
of the groups $G$ arising in this way next time. A reference for this
material is Artin's book {\sl Algebra}, on reserve (and in the stacks)
of the Math Library.

\end{document}
