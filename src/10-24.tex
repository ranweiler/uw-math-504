\documentclass[10pt]{article}
\usepackage{amsmath, amssymb}

\begin{document}

\section*{Math 504, 10/24}

Last time we explored the functor Hom$_R(M,--)$ sending any $R$-module
$N$ to the set of $R$-linear maps from $M$ to $N$. We saw that this
functor preserves short exact sequences (and in fact exact sequences
generally) if and only if $M$ is a projective $R$-module; in turn this
holds if and only if $M$ is a direct summand of a free $R$-module. We
now look at the functor Hom$_R(--,M)$ obtained by fixing $M$ as the
range rather than the domain. The first thing to note is that an
$R$-module map $f:N\rightarrow P$ now gives rise to map
Hom$_R(P,M)-\rightarrow\,$Hom$_R(N,M)$ in the {\sl other direction},
from $P$ to $N$, by composition with $f$; accordingly we say that the
functor Hom$_R(--,M)$ is {\sl contravariant}. Given a short exact
sequence $0\rightarrow A\rightarrow B\rightarrow C\rightarrow 0$ we find
that the induced sequence
$0\rightarrow\,$Hom$_R(C,M)\rightarrow\,$Hom$_R(B,M)\rightarrow\,$Hom$_R(A,M)$
is exact (for example, any nonzero map from $C$ to $M$ pulls back to a
nonzero one from $B$ to $M$, since the map from $B$ to $C$ is
surjective), but again, as with Hom$_R((M,--)$, there can be trouble on
the right-hand end: not every map from $A$ to $M$ extends to a map from
$B$ to $M$. If every map from $A$ to $M$ does extend to $B$, for any
choice of $R$-modules $A,B$ with $A$ injecting into $B$, then we call
$M$ {\sl injective}.

Unfortunately it is considerably more difficult to decide which
$R$-modules are injective than it was for projective modules. Consider
first a special case. Suppose that the module $M$ is such that every
$R$-module map from an {\sl ideal} $I$ of $R$ into $M$ extends to a map
from $R$ into $M$. Let $A$ be a submodule of $B$ and suppose we have a
map $f$ from $A$ into $M$ that we want to extend to $B$. Pick $x\in
B,x\notin A$. The set of all $r\in R$ with $rx\in A$ is an ideal $I$ of
$R$ and we have a map from $I$ to $M$ sending $i\in I$ to $f(ix)$.
Extending this map to $R$, we now have an extension of $f$ to all of
$Rx$, which by linearity extends $f$ to the submodule $A+Rx$ properly
containing $A$. Iterating this process many times (this requires the
axiom of choice, to be discussed later), we extend $f$ to all of $B$, as
desired. So we are reduced to asking when maps $f$ from ideals $I$ of
$R$ to $M$ extend to $R$. This question is easiest to answer whence $R$
is a PID, for then every ideal $I$ is principal. If $I=(i)$, then
extending $f$ to $R$ amounts to deciding what $f(1)$ should be; the
requirement on $f(1)$ is exactly that $if(1) = f(i)$. Thus in this case
the key property of $M$ that we need is that {\sl for every nonzero
  $i\in R$ and $m\in M$, there is $m_i\in M$ with $im_i = m$ (note that
  this property arose in connection with the problem in the last
  homework set to show that the direct product of countably many copies
  of $\mathbf{Z}$ is not free over $\mathbf{Z}$. Note also that {\sl
    every} $m\in M$ must satisfy this requirement, since given any $m\in
  M$ and nonzero $i\in R$, we get a map from the principal ideal $(i)$
  of $R$ to $M$ sending any multiple $ri$ to $rm$.) We say that $M$ is
  {\sl divisible} if it has this property. Thus {\sl over a PID, a
    module is injective if and only if it is divisible}. We are forced
  to specialize to PIDs to get an easy criterion for injectivity; for
  projectivity, by contrast, our results apply over any commutative
  ring, In homework for this week you will show that any
  $\mathbf{Z}$-module $M$ injects into an injective one. We will later
  generalize this result to modules over any ring $R$, after we
  introduce the analogue of projectivity and injectivity for tensor
  products and establish a relation between tensor products and
  homomorphisms. The corresponding result for projective modules is that
  {\sl every $R$-module (for any $R$) is a quotient of a projective
    one}; this is easy to see since we already know that any $R$-module
  is a quotient of a free one and free modules are projective.

Divisible modules over integral domains are interesting in their own
right, furnishing a stark contrast to finitely generated modules over
PIDs. It is obvious for example that $\mathbf{Q}$ is a divisible
$\mathbf{Z}$-module, but so is the quotient $\mathbf{Q}/\mathbf{Z}$
(this quotient is a $\mathbf{Z}$-module, not a ring; in particular, an
injective $\mathbf{Z}$-module is {\sl not} necessarily a
$\mathbf{Q}$-module,) More generally, any quotient of a divisible
$R$-module is again divisible (but a quotient of an injective module
need not be injective in general). For a more exotic example fix a prime
number $p$ and look at $\mathbf{Z}[1/p]/\mathbf{Z}$; here
$\mathbf{Z}[1/p]$ is the subring of $\mathbf{Q}$ generated by
$\mathbf{Z}$ and $1/p$. This last quotient is isomorphic to
$\mathbf{Q}/\mathbf{Z}_{(p)}$, where $\mathbf{Z}_{(p)}$ consists of all
$a/b\in\mathbf{Q}$ with $p$ not dividing the integer $b$. We will see
more of $\mathbf{Z}_{(p)}$, which is a ring as well as a
$\mathbf{Z}$-module, in the spring.

\end{document}
