\documentclass[10pt]{article}
\usepackage{amsmath, amssymb}

\begin{document}

\section*{Math 504, 10/28}

We now point out a fundamental connection between tensor products and
homomorphisms, which indicates why the tensor product functor is right
exact while the Hom functions are left exact. Assume first that $R$ is
commutative and that $A,B,C$ are $R$-modules. Then {\sl there is an
  isomorphism}

Hom$_R(B\otimes_R A,C)\cong\,$Hom$_R(A,$Hom$_R(B,C))$

{\sl sending a homomorphism $f$ on the left to the homomorphism mapping
  $a\in A$ to the map $g$ defined by $g(b) = f(b\otimes a)$}. We express
this relationship by saying that {\sl the tensor product functor is the
  left adjoint of the Hom functor, or that the Hom functor is the right
  adjoint of the tensor product functor}, since moving across the comma
converts Hom to $\otimes$. We need an extension of this to show that
every left module over a possibly noncommutative ring $R$ embeds in an
injective one. Let $S$ be another ring, let $A$ be a left $R$-module,
let $C$ be a left $S$-module, and finally let $B$ be an $(S,R)$
bimodule, so that $B$ is simuntaneously a left $S$-module and a right
$R$-module and we have $(sb)r = s(br)$ for all $r\in R,s\in S,b\in B$.
As noted in the last lecture,'/ Hom$_R(B,C)$ is then a left $R$-module
(making $R$ act on the domain rather than the range). Now our extension
reads

Hom$_S(B\otimes_R A,C)\cong\,$Hom$_R(A,$Hom$_S(B,C))$

and again a homomorphism $f$ on the left is sent to the homomorphism
taking $a\in A$ to the map sending $b\in B$ to $f(b\otimes a)$. We apply
this map in the special case where our ring $S$ is the integers $\mathbf
Z, B=R, A=N$ is a left $R$-module, and $C=Q$ is an injective $\mathbf
Z$-module containing the left $R$-module $M$ (whose existence you showed
in homework this week). Then we have

Hom$_{\mathbf Z}(R\otimes_R N,Q) \cong\,$Hom$_{\mathbf
  Z}(N,Q)\cong\,$Hom$_R(N,$Hom$_{\mathbf Z} (R,Q))$

and since the functor taking $N$ to Hom$_{\mathbf Z}(N,Q)$ is exact, so
too is the functor taking $N$ to Hom$_R(N,$Hom$_{\mathbf Z}(R,Q))$.
Hence Hom$_{\mathbf Z}(R,Q)$ is an injective left $R$-module. It
contains Hom$_R(R,M)\cong M$, as desired. Whew! This argument is a real
workout in manipulating the formalism surrounding Hom and the tensor
product, but it is well worth studying in detail, as ultimately such
formalism is much more efficient than applying the definition of
injective module directly. To summarize again, it shows that {\sl any
  left module over any ring $R$ embeds in an injective one}. Given a
left $R$-module $M$, there is in fact in a precise sense a unique
smallest injective $R$-module containing $M$, called its {\sl injective
  hull}, but I will not pursue its construction here (you can read about
it in \S10.5 of Dummit and Foote). The corresponding object on the
projective side is called the {\sl projective cover}, but it does not
exist in the category of general $R$-modules, though it does in certain
restricted subcategories of this one. Another example of the interplay
between projectivity and injectivity is the observation that {\sl direct
  sums of porjective modules are projective, while direct products of
  injective modules are injective}; this holds because given any
collection of $R$-modules $M_i$ and another $R$-module $M$, any set of
homomorphisms from $M_i$ to $M$ (one for each $i$) induces a unique
homomorphism from the direct sum of the $M_i$ to $M$, while any set of
homomorphisms from $M$ to $M_i$ induces one from $M$ to the direct
product of the $M_i$. In the language of category theory, direct sums
are called products, while direct products are called coproducts.

\end{document}
