\documentclass[10pt]{article}
\usepackage{amsmath, amssymb}

\begin{document}

\section*{Math 504, 10/19}

Continuing with tensor products, let us return for a moment to finitely
generated modules $M$ over a PID $R$ and show how we can use tensor
products to recover the uniqueness of the decomposition of $M$ as a
direct sum of quotients of $R$. Indeed, let $K$ be the field of
fractions of $R$, consisting by definition of all formal fractions $a/b$
with $a,b\in R,b\ne0$. Then $K\otimes_R R/(d) = 0$ for all $d\ne0$ in
$R$: any tensor $k\otimes x = (k/d)d \otimes x = (k/d)\otimes dx = 0$.
On the other hand, $K\otimes_R R\cong K$: the isomorphism sends
$k\otimes r$ to $kr$ (the map sending $(k,r)$ to $kr$ is clearly
bilinear out of $K\times R$, and induces an isomorphism from the tensor
product to $K$. Thus tensoring $M$ with $K$ replaces each copy of $R$ in
$M$ with $K$, so the number of copies of $R$ in $M$ cannot depend on the
choice of decomposition of $M$. As for the torsion submodule $T$ of $M$,
it too can be analyzed via tensor products instead of by quotients as we
did in class last week: $R/(d)\otimes_R R/(e)\cong R/(c)$, where $c$ is
a gcd of $d$ and $e$ (you will prove a special case of this in
homework), whence by working with various powers $p^k$ of irreducible
elements in $R$ you can show that the number of copies of $R/(p^k)$
occurring in $M$, or in its torsion submodule, is independent of the
choice of decomposition of $M$.

The construction of the tensor product extends in a natural way to any
finite set $M_1,\ldots,M_k$ of $R$-modules: multinear maps from
$M_1\times\cdots\times M_k$ to an $R$-module $P$ correspond bijectively
to $R$-linear maps from the tensor product $M_1\otimes\cdots\otimes M_k$
to $P$. Here I leave the definitions of \lq\lq multilinear" and tensor
product in this setting to you. In the special case of copies of the
{\sl same} module $M$ there are a couple of important related
constructions called symmetric and exterior powers. A bilinear map $f$
from $M\times M$ to $P$ is called {\sl symmetric} if $f(m,n) = f(n,m)$
for all $m,n\in M$. Such maps are in bijection to linear maps, not from
the tensor product $M\otimes M$ to $P$, but from the quotient of it by
the submodule generated by all differences $m\otimes n - n\otimes m$ as
$m,n$ range over $M$; the resulting quotient is denoted $S^2 M$ and is
called the {\sl symmetric square} of $M$. The elements of $S^2 M$ are
usually denoted in the same way as elements of $M\otimes M$ (the latter
is sometimes denoted $T^2 M$ and called the {\sl tensor square} of $M$),
but it is understood that an element $m\otimes n$ of $S^2 M$, called a
symmetric tensor, is identified with $n\otimes m$. In a similar manner,
higher symmetric powers $S^k M$ are defined as quotients of $T^k M =
M\otimes\cdots\otimes M$ by the submodule generated by suitable
differences and the elements of $S^k M$ are again called symmetric
tensors. A similar but often more useful construction results if we
instead look at {\sl alternating} bilinear maps $f$ from $M\times M$ to
$P$, satisfying $f(m,m) = 0, f(m,n) = -f(n,m)$ for all $m,n\in M$. Here
one should replace $S^2 M$ by the quotient of $M\otimes M$ by the
submodule generated by all sums $m\otimes n + n\otimes m$ and tensors
$m\otimes m$; the resulting module is denoted by $\bigwedge^2 M$. In a
similar way one defines alternating multilinear maps from
$M\times\cdots\times M$ to $P$ and there is a bijection between them and
linear maps from the {\sl exterior power} $\bigwedge^k M$ to $P$. As for
tensor products, we find that both symmetric and exterior powers of free
$R$-modules are free, but this time their ranks behave very differently.
In fact, if $M$ is free of rank $n$, then in particular $\bigwedge^n M$
is free of rank 1 and $\bigwedge^k M = 0$ if $k>n$! (You will work out
the formula for the rank of an arbitrary power $\bigwedge^k R^n$ in
homework for next week and exhibit a basis for it.) That $\bigwedge^n
R^n\cong R$ is often expressed by saying that {\sl the determinant is
  the only alternating multilinear function from the columns of an
  $n\times n$ matrix over $R$ to $R$, up to scalar multiple}.

Taking the direct sum of all the tensor powers of $M$ (with the $0$th
power being by definition the base ring $R$) we get the {\sl tensor
  algebra $TM$ of $M$}, so called because it admits a natural
multiplication, taking the product of $x_1\otimes\cdots\otimes x_n$ and
$y_1\otimes\cdots\otimes y_m$ to be $x_1\otimes\cdots\otimes x_n\otimes
y_1\otimes\cdots\otimes y_n$. In a similar manner one obtains the {\sl
  symmetric} and {\sl exterior} algebras of $M$, denoted respectively by
$SM$ and $\bigwedge M$. If $R=K$ is a field and $M$ is
finite-dimensional over it, say with basis $x_1,\ldots,x_n$, then the
symmetric algebra $SM$ may be identified in a natural way with the ring
of polynomials $K[x_1,\ldots,x_n]$ over $K$ in the variables
$x_1,\ldots,x_n$, while the exterior algebra $\bigwedge M$ is
finite-dimensional over $K$, in fact of dimension $2^n$.

\end{document}
