\documentclass[10pt]{article}
\usepackage{amsmath, amssymb}

\begin{document}

\section*{Math 504, 10/26}

We now address in more detail the infinitely many choices we had to make
to prove a crucial result last time (that an $R$-module $M$ is injective
if and only if every $R$-module map from an ideal $I$ of $R$ into $M$
extends to $R$). The most convenient setting for discussing infinitely
many choices has proved to be the following one. Let $P$ be a nonempty
partially ordered set, or poset, so that $P$ has an order relation $\le$
among its elements such that $x\le x, x\le y$ and $y\le x$ imply that
$x=y$, and $x\le y$ and $y\le z$ imply that $x\le z$, for all $x,y,z\in
P$. We do not require for every $x,y\in P$ that $x\le y$ or $y\le x$;
that is why we call $P$ partially (rather than totally) ordered. Now
suppose that every totally ordered subset $C$ of $P$ has an upper bound
$z$, so that $x\le z$ for all $x\in C$ (the element $z$ may or may not
lie in $C$). Then {\sl Zorn's Lemma} asserts that $P$ has a maximal
element $y$ (such that the only $x\in P$ with $y\le x$ is $y$ itself).
This principle finds wide application in algebra; in our present
setting, let $M$ be an $R$-module such that every $R$-module map from an
ideal $I$ of $R$ to $M$ extends to a map from $R$ to $M$ and let
$A\subset B$ be $R$-modules and $f$ an $R$-module map from $A$ to $M$.
Consider the set of ordered pairs $(A',f')$ where $A'$ is a submodule of
$B$ containing $A$ and $f'$ is an $R$-module map extending $f$ on $A$.
This set is partially ordered by inclusion and every totally ordered
subset has an upper bound (given by taking the union of the submodules
and maps in question), so it has a maximal element $(\tilde A,\tilde
f)$, which by the first part of the argument last time must be defined
on all of $B$. Thus $M$ is injective over $R$, as desired.

So far we have studied the functors Hom$_R(M,\text{--})$ and
Hom$_R(\text{--},M)$ in some detail, for a fixed $R$-module $M$; the
first is covariant and the second is contravariant, while both are left
but not right exact in general. We now look at the other operation we
have learned this term, namely the tensor product. The functor
$M\otimes_R \text{--}$ is a covariant one for any fixed $R$-module $M$:
any map $f$ from $N$ to $P$ induces one from $M\otimes_R N$ to
$M\otimes_R P$, sending a tensor $m\otimes n$ to $m\otimes f(n)$. This
time this functor is right but not left exact; if we have a short exact
sequence $0\rightarrow A\rightarrow B\rightarrow C\rightarrow 0$, then
the induced map from $M\otimes_R B$ to $M\otimes_R C$ is surjective
(since the tensors $m\otimes c$ for $c\in C$ generate $M\otimes_R C$ as
an abelian group), and has kernel exactly the image of $M\otimes_R A$ in
$M\otimes_R B$ (since an explicit inverse to the map from $M\otimes_R B$
modulo this image to $M\otimes_R C$ is given by mapping a tensor
$m\otimes c$ to $m\otimes b$, where $b$ is any preimage of $c$ for the
map from $B$ to $C$), but the map from $M\otimes_R A$ to $M\otimes_R B$
need not be injective. The simplest example, analogous to one given
earlier for projectivity, has $R=\Bbb Z, M=\Bbb Z_2,A=B=\Bbb Z,C=\Bbb
Z_2$ with the map from $A$ to $B$ sending an integer $x$ to $2x$. Here
the induced map from $M\otimes A$ to $M\otimes B$ sends $m\times a$ to
$m\otimes 2a = 2m\otimes a = 0$, so this map is 0. The notion
corresponding to projectivity or injectivity in this setting is called
{\sl flatness}: an $R$-module $M$ is flat if given any injective map
$f:A\rightarrow B$ of $R$-modules, the induced map $1\otimes f$ sending
$m\otimes a$ to $m\otimes f(a)$ is injective. Fortunately there is a
close connection between projectivity and flatness: {\sl any projective
  $R$-module is flat}. To see this first note that tensor products
commute with direct sums; the tensor product $(M\oplus N)\otimes_R P$ is
isomorphic to $M\otimes_R P\oplus N\otimes P$, since a bilinear map from
$(M\oplus N) \times P$ to another $R$-module is completely determined by
its restrictions to $M\times P$ and $N\times P$, which must be bilinear.
Thus tensoring an $R$-module $M$ with a free $R$-module amounts to
replacing $M$ by the sum of $r$ copies of itself, $r$ the rank of the
free module (since $R\otimes_R M\cong M$) and so is exact; similarly
tensoring $R$-modules with a direct summand of a free module is exact.
For a different example, {\sl $\Bbb Q$ is a flat $\Bbb Z$-module}. To
see this, we need to give a different construction of $\Bbb
Q\otimes_{\Bbb Z} M$ for a $\Bbb Z$-module $M$, generalizing the
construction of the field of fractions of an integral domain. Given $M$,
let $M'=\Bbb Z*^{-1} M$, the {\sl localization of $M$ at $\Bbb Z*$,
  consist by definition of all ordered pairs $(u,m)$ with $u\in\Bbb Z*,
  m\in M$. Identify any two pairs $(u_1,m_1)$ and $(u_2,m_2)$ whenever
  there is a nonzero integer u with $u(u_1 m_2 - u_2 m_1) = 0$. One
  checks that this relation is reflexive, symmetric and transitive, so
  is an equivalence relation; the proof of transitivity requires the
  element $u$ and would not work if the condition for $(u_1,m_1) =
  (u_2,m_2)$ were just $u_1 m_2 - u_2 m_1 = 0$. Then one checks that
  $M'$ is isomorphic to $\Bbb Q\otimes_{\Bbb Z} M$, the isomorphism
  sending $(u,m)$ to $u^{-1}\otimes m$. Now let $M\subset N$ be any
  $\Bbb Z$-modules. The construction of the tensor product shows that
  any element of $\Bbb Q\otimes_{\Bbb Z} M$ may be written as
  $(1/d)\otimes m$ for some $m\in M$. If the induced map from $\Bbb
  Q\otimes_{\Bbb Z} M$ to $\Bbb Q\otimes_{\Bbb Z} N$ had a nonzero
  kernel $K$, containing say $(1/d)\otimes m$, then our localization
  construction would show that $cm = 0$ for some $c\in\Bbb Z*$; but then
  $(1/d)\otimes m$ is already 0 in $\Bbb Q\otimes_{\Bbb Z} M$. Hence
  $K=0$, as claimed. More generally, {\sl the field of fractions of any
    integral domain $R$ is a flat $R$-module}.

So far we have worked with modules over a commutative ring $R$.
Everything that we did can be duplicated word for word for left modules
over any ring $R$ (with 1); the only caveat is the set Hom$_R(M,N)$ of
$R$-module maps between any two such modules $M,N$ is only an abelian
group, not an $R$-module. Similarly, but more subtly, we can form the
tensor product $M\otimes_R N$ of a {\sl right} $R$-module $M$ with a
{\sl left} one $N$, identifying $mr\otimes n$ with $m\otimes rn$ in the
tensor product; such a tensor product is neither a right nor a left
$R$-module in general, but it is still an abelian group. Now let $R,S$
be rings and $M,N$ left $R$-modules such that $R$ is an
$(R,S)$-bimodule, admitting an $S$-module structure on the right in
addition to its $R$-module structure on the left and also satisfying
$(rm)s = r(ms)$ for all $m\in M,r\in R,s\in S$. Then Hom$_R(M,N)$ admits
a {\sl left} $S$-module structure, defined via $s\cdot f(m) = f(ms)$
(whether or not $N$ has an $S$-module structure). We will use this
structure to show that any left $R$-module embeds into an injective one.

\end{document}
